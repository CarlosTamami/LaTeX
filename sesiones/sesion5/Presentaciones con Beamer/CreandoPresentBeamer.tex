% $Header: /cvsroot/latex-beamer/latex-beamer/solutions/conference-talks/conference-ornate-20min.en.tex,v 1.6 2004/10/07 20:53:08 tantau Exp $

\documentclass[11pt]{beamer}
\usepackage[spanish]{babel}
\usepackage{amsthm}





\mode<presentation> {
  \usetheme{Warsaw}
  % or ...

  \setbeamercovered{transparent}
  % or whatever (possibly just delete it)
}
\usefonttheme[stillsansseriftext,stillsansserifsmall,stillsansseriflarge]{serif}
%\usefonttheme{structuresamallcapsserif}

%\usepackage[spanish]{babel}
% or whatever

\usecolortheme{whale}

\usepackage[latin1]{inputenc}
% or whatever

\usepackage{graphicx,color}

\usepackage{times}
\usepackage[T1]{fontenc}
% Or whatever. Note that the encoding and the font should match. If T1
% does not look nice, try deleting the line with the fontenc.

\font\Bb=msbm10

%\colorlet{structure}{green!60!black}
\def\colorize<#1>{%
\temporal<#1>{\color{red!50}}{\color{black}}{\color{verde1!50}}}


\useinnertheme{rounded}
\useoutertheme{infolines}
%\useoutertheme{sidebar}
%\usefonttheme{structuresmallcapsserif}


\colorlet{verde1}{green!80!gray} \colorlet{verde2}{green!60!gray}
\colorlet{verde3}{green!40!gray} \colorlet{verde4}{-green}

%\renewcommand{\theequation}{\thesection.\arabic{equation}}

\setbeamercolor{title}{fg=red!80!black,bg=red!20!white}
\setbeamertemplate{blocks}[rounded][shadow=true]
\setbeamertemplate{section}[circle]



\colorlet{mystruct}{structure} % Save current structure
\colorlet{structure}{magenta} % New structure
\usestructuretemplate{\color{structure}}{} % \structure{..}
\beamertemplateshadingbackground{yellow!50}{blue!50} % New background

\colorlet{structure}{mystruct}
\beamertemplateshadingbackground{blue!10}{yellow!10}


\title[Beamer]{ Creando
presentaciones con Beamer\\Edici\'{o}n de Textos Cient\'{\i}ficos
con \LaTeX\\Nivel II}


%\setbeamertemplate{title page}{
%\includegraphics[height=\paperheight,width=\paperwidth]{carne1.jpg}
%}



\author[L\'opez-Mart\'inez-Quesada]{A.J. L\'{o}pez  \and Juan Mart\'{\i}nez Moreno \and
J.M. Quesada }
% - Give the names in the same order as the appear in the paper.
% - Use the \inst{?} command only if the authors have different
%   affiliation.

\institute[Ja\'{e}n] % (optional, but mostly needed)
{
  Universidad de Ja\'{e}n\\
  Departamento de Matem\'{a}ticas
}
% - Use the \inst command only if there are several affiliations.
% - Keep it simple, no one is interested in your street address.

\date[Innovaci\'{o}n Docente, 2007] % (optional, should be abbreviation of conference name)
{Cursos de Innovaci\'{o}n Docente, 2007}
% - Either use conference name or its abbreviation.
% - Not really informative to the audience, more for people (including
%   yourself) who are reading the slides online

\subject{}

\beamertemplatetransparentcoveredhigh

% Delete this, if you do not want the table of contents to pop up at
% the beginning of each subsection:

\AtBeginSection[] {
  \begin{frame}<beamer>
    \frametitle{\'{I}ndice}
    \tableofcontents[currentsection]
  \end{frame}
}

\AtBeginSubsection[] {
  \begin{frame}<beamer>
    \frametitle{\'{I}ndice}
    \tableofcontents[currentsection,currentsubsection]
  \end{frame}
}


% If you wish to uncover everything in a step-wise fashion, uncomment
% the following command:

%\beamerdefaultoverlayspecification{<+->}








\begin{document}


%\setbeamerfont{title}{shape=\itshape,family=\rmfamily}



\begin{frame}
  \titlepage
\end{frame}

\begin{frame}
  \frametitle{\'{I}ndice}
  \tableofcontents[pausesections]
  % You might wish to add the option [pausesections]
\end{frame}



\section{Creando presentaciones}

\begin{frame}

\begin{itemize}
\item Existe un excelente paquete en \LaTeX ~para crear presentaciones: \alert{Beamer}.

\item Beamer est\'a incluido en MikTex.

\item Beamer est\'a dise\~nado para generar directamente un PDF via
PDFLatex.

\end{itemize}

\end{frame}

\section{Estructura b\'asica}

\begin{frame}[fragile]

\begin{itemize}
\item El formato b\'asico de un texto en Beamer es como sigue:
\begin{block}{} \scriptsize
\begin{semiverbatim}\alert<1>{\\documentclass\{beamer\}}
\alert<2>{\\title[T\'itulo corto]\{T\'itulo largo\}}
\alert<3>{\\subtitle\{...\} % Opcional}
\alert<4>{\\author\{...\}}
\alert<5>{\\institute\{...\} % Opcional}
\alert<6>{\\begin\{document\}}
\alert<7>{\\begin\{frame\}}
\alert<8>{\\titlepage}
\alert<9>{\\end\{frame\}}
\alert<10>{%Trasparencias}
\alert<11>{\\end\{document\}}

\end{semiverbatim}
\end{block}
\end{itemize}

\end{frame}
\normalsize



\begin{frame}[fragile]

\begin{itemize}[<+-| alert@+>]
\item Las transparencias est\'an escritas en el entorno \alert{frame}.
\item Cualquier comando en \LaTeX ~puede emplearse.

\item
Transparencia t\'ipica: \begin{block}{} \scriptsize
\begin{semiverbatim}
\\begin\{frame\}
\\frametitle\{T\'itulo de la trasparencia\}
\\begin\{itemize\}
\\item Punto 1
\\item Punto 2
\\item Punto 3
\\end\{itemize\}
\\end\{frame\}
\end{semiverbatim}
\end{block}
\end{itemize}

\end{frame}




\begin{frame}
\begin{block}{Ejercicio}
\begin{enumerate}
\item Crea un fichero ejemplo.
\item Introduce informaci\'on de autor/t\'itulo en el pre\'ambulo.
\item Escribe una o varias trasparencias
\item Crea el PDF y visi\'onalo.
\end{enumerate}
\end{block}
\end{frame}


\section{Estilos}

\begin{frame}\frametitle{Temas de presentaci\'on}


\begin{block}{} \scriptsize
\begin{semiverbatim}\\usetheme[opciones]\{nombre de tema\}:
\end{semiverbatim}
\end{block}



\begin{itemize}
\colorize<1>\item Temas antiguos bars, boxes, classic, default,
lined, plain, shadow, sidebar, sidebardark, sidebardarktab,
sidebartab, split, tree, treebars

\colorize<2>\item Temas con navegaci\'on: default, boxes, Bergen,
Madrid, Pittsburgh, Rochester

\colorize<3>\item Temas en \'arbol: Antibes, JuanLesPins,
Montpellier.

\colorize<4>\item Temas con TOC: Berkeley, PaloAlto, Goettingen,
Marburg, Hannover

\colorize<5>\item Temas con mini navegaci\'on: Berlin, Ilmenau,
Dresden, Darmstadt, Frankfurt, Singapore, Szeged

\colorize<6>\item Temas con t\'itulos y subt\'itulos: Copenhagen,
Luebeck, Malmoe, Warsaw
\end{itemize}

\begin{block}<7->{} \scriptsize
\begin{semiverbatim}

\\usetheme\{Warsaw\}
\end{semiverbatim}
\end{block}
\end{frame}



\begin{frame}\frametitle{Colores y fuentes de los temas}

\begin{block}{} \scriptsize
\begin{semiverbatim}\\usecolortheme[opci\'on]\{nombre\}
\end{semiverbatim}
\end{block}

\begin{itemize}
\item Temas completos de color: albatross, beetle, crane, dove, fly,
seagull

\item Temas de color interior: lily, orchid

\item Temas de color exterior: whale, seahorse

\end{itemize}



\begin{block}{} \scriptsize
\begin{semiverbatim}\\usefonttheme[opci\'on]\{nombre\}
\end{semiverbatim}
\end{block}

\begin{itemize}
\item fuentes: default, professionalfonts, serif, structurebold,
structureitalicserif, structuresmallcapsserif
\end{itemize}
\end{frame}



\section{Colores}

\begin{frame}
\begin{itemize}[<+-| alert@+>]
\item Beamer carga el paquete \texttt{xcolor}

\begin{block}{} \scriptsize
\begin{semiverbatim}
\\xdefinecolor\{lavanda\}\{rgb\}\{0.8,0.6,1\}

\\xdefinecolor\{oliva\}\{cmyk\}\{0.64,0,0.95,0.4\}
\end{semiverbatim}
\end{block}

\begin{block}{} \scriptsize
\begin{semiverbatim}
\\colorlet\{structure\}\{green!60!black\}
\end{semiverbatim}
Para la sustituci\'on de color
\end{block}

\item Colores Predefinidos: red, green, blue,
cyan, magenta, yellow, black, darkgray, gray, lightgray, orange,
violet, purple y brown.



\item La mezcla de colores es muy sencilla
\begin{center}
\begin{tabular}{ccc}
\hline
  green!80!gray & {\color{verde1} texto} & 80\% verde + 20\% gris \\
  green!60!gray & {\color{verde2} texto} & 60\% verde + 40\% gris \\
  green!40!gray & {\color{verde3} texto} &  40\% verde + 60\% gris\\
  -green & {\color{verde4} texto} & quita el verde \\
  \hline
\end{tabular}
\end{center}

\end{itemize}
\end{frame}


\begin{frame}

\begin{block}{} \scriptsize
\begin{itemize}
\item \begin{semiverbatim}\\alert\{texto\} \end{semiverbatim} $\Longrightarrow$ \alert{texto}
\item \begin{semiverbatim}\\structure\{texto\}\end{semiverbatim} $\Longrightarrow$ \structure{texto}
\end{itemize}
\end{block}


\begin{block}{} \scriptsize
\begin{itemize}
\item Color s\'olido de fondo \begin{semiverbatim}
\\beamersetaveragebackground\{color\}

\\beamertemplatesolidbackgroundcolor\{color\}
\end{semiverbatim}

\item Color de fondo gradual \begin{semiverbatim}
\\beamertemplateshadingbackground\{color1\}\{color2\}
\end{semiverbatim}

\end{itemize}
\end{block}

\end{frame}



\colorlet{mystruct}{structure} % Save current structure
\colorlet{structure}{magenta} % New structure
\usestructuretemplate{\color{structure}}{} % \structure{..}
\beamertemplateshadingbackground{yellow!50}{blue!50} % New background
\begin{frame}

\begin{block}{El color de esta p\'agina}
\begin{semiverbatim}\scriptsize
\\colorlet\{mystruct\}\{structure\} \% Guarda la estructura actual

\\colorlet\{structure\}\{magenta\} \% Nueva estructura

\\usestructuretemplate\{\\color\{structure\}\}\{\} % \structure{..}

\\beamertemplateshadingbackground\{yellow!50\}\{blue!50\} % New background
\end{semiverbatim}
\end{block}

\begin{block}<2->{Volvemos al color antiguo}
\begin{semiverbatim}\scriptsize
\\colorlet\{structure\}\{mystruct\}

\\beamertemplateshadingbackground\{blue!10\}\{yellow!10\}
\end{semiverbatim}
\end{block}

\end{frame}
\colorlet{structure}{mystruct}
\beamertemplateshadingbackground{blue!10}{yellow!10}


\section{Teoremas y otros entornos}

\begin{frame}


\begin{block}{}
\begin{semiverbatim}\scriptsize
\\begin\{block\}\{\}

Un  \\alert\{conjunto\} consiste de elementos.

\\end\{block\}
\end{semiverbatim}
\end{block}

\begin{block}{}
Un \alert{conjunto} consiste de elementos.
\end{block}





\begin{block}<2->{}
\begin{semiverbatim}\scriptsize
\\begin\{block\}<2->\{Def.\}

Un  \\alert\{conjunto\} consiste de elementos.

\\end\{block\}
\end{semiverbatim}
\end{block}

\begin{block}<2->{Def.}
Un \alert{conjunto} consiste de elementos.
\end{block}


\begin{block}<3->{}
\begin{semiverbatim}\scriptsize
\\begin\{alertblock\}\{Def.\}

Un  \\alert\{conjunto\} consiste de elementos.

\\end\{alertblock\}
\end{semiverbatim}
\end{block}

\begin{alertblock}<3->{Def.}
Un \alert{conjunto} consiste de elementos.
\end{alertblock}\end{frame}



\begin{frame}


\begin{block}{}
\begin{semiverbatim}\scriptsize
\\begin\{theorem\}

There exists an infinite set.

\\end\{theorem\}
\end{semiverbatim}
\end{block}


\begin{theorem}<1->
There exists an infinite set.
\end{theorem}

\begin{block}<2->{}
\begin{semiverbatim}\scriptsize

\\begin\{proof\}

This follows from the axiom of infinity.

\\end\{proof\}
\end{semiverbatim}
\end{block}


\begin{proof}<2->
This follows from the axiom of infinity.
\end{proof}

\begin{block}<3->{}
\begin{semiverbatim}\scriptsize

\\begin\{example\}[Natural Numbers]

The set of natural numbers is infinite.

\\end\{example\}
\end{semiverbatim}
\end{block}


\begin{example}<3->[Natural Numbers]
The set of natural numbers is infinite.
\end{example}
\end{frame}

\section{Columnas}

\begin{frame}
\frametitle{Columnas}


\begin{block}{}
\begin{semiverbatim}\scriptsize
\\begin\{columns\}

\\column\{.45\\textwidth\}

\\begin\{block\}\{Pregunta\}

$x^2=1$?

\\end\{block\}

\\column\{.45\\textwidth\}

\\begin\{block\}\{Respuesta\}

$x=1$

\\end\{block\}

\\end\{columns\}
\end{semiverbatim}
\end{block}

\begin{columns}
\column{.45\textwidth}
\begin{block}{Pregunta}
$x^2=1$?
\end{block}
\column{.45\textwidth}
\begin{block}{Respuesta}
$x=1$
\end{block}
\end{columns}
\end{frame}


\section{Presentaciones din\'amicas}

\begin{frame}

\begin{block}{}

\begin{columns}
\column{.45\textwidth}
\begin{semiverbatim}\scriptsize
\\begin\{itemize\}
\end{semiverbatim}

\begin{semiverbatim}\scriptsize
\\pause \\item Lo
\end{semiverbatim}

\begin{semiverbatim}\scriptsize
\\pause \\item que
\end{semiverbatim}

\begin{semiverbatim}\scriptsize
\\pause \\item empieza
\end{semiverbatim}

\begin{semiverbatim}\scriptsize
\\pause \\item acaba.
\end{semiverbatim}

\begin{semiverbatim}\scriptsize
\\end\{itemize\}
\end{semiverbatim}



\column{.45\textwidth}
\begin{itemize}
\pause \item Lo  \pause \item que \pause \item empieza \pause
\item acaba.
\end{itemize}


\end{columns}
\end{block}



\end{frame}


\begin{frame}

\begin{block}{}

\begin{columns}
\column{.45\textwidth}
\begin{semiverbatim}\scriptsize
\\begin\{itemize\}
\end{semiverbatim}

\begin{semiverbatim}\scriptsize
 \\item Lo
\end{semiverbatim}

\begin{semiverbatim}\scriptsize
 \\item<2-> que
\end{semiverbatim}

\begin{semiverbatim}\scriptsize
 \\item<3-> empieza
\end{semiverbatim}

\begin{semiverbatim}\scriptsize
 \\item<4-> acaba.
\end{semiverbatim}

\begin{semiverbatim}\scriptsize
\\end\{itemize\}
\end{semiverbatim}



\column{.45\textwidth}
\begin{itemize}
\pause \item Lo  \pause \item que \pause \item empieza \pause
\item acaba.
\end{itemize}


\end{columns}
\end{block}






\end{frame}




\begin{frame}

\begin{block}{}

\begin{columns}
\column{.45\textwidth}
\begin{semiverbatim}\scriptsize
\\begin\{itemize\}[<+->]
\end{semiverbatim}

\begin{semiverbatim}\scriptsize
 \\item Lo
\end{semiverbatim}

\begin{semiverbatim}\scriptsize
 \\item que
\end{semiverbatim}

\begin{semiverbatim}\scriptsize
 \\item empieza
\end{semiverbatim}

\begin{semiverbatim}\scriptsize
 \\item ...
\end{semiverbatim}

\begin{semiverbatim}\scriptsize
\\end\{itemize\}
\end{semiverbatim}



\column{.45\textwidth}
\begin{itemize}
\pause \item Lo  \pause \item que \pause \item empieza \pause
\item ...
\end{itemize}


\end{columns}
\end{block}






\end{frame}


\begin{frame}

\begin{block}{}

\begin{columns}
\column{.45\textwidth}
\begin{semiverbatim}\scriptsize
\\begin\{itemize\}
\end{semiverbatim}

\begin{semiverbatim}\scriptsize
 \\item<1-> Lo
\end{semiverbatim}

\begin{semiverbatim}\scriptsize
 \\item<3-4> que
\end{semiverbatim}

\begin{semiverbatim}\scriptsize
 \\item<3-> empieza
\end{semiverbatim}

\begin{semiverbatim}\scriptsize
 \\item<2-4> acaba.
\end{semiverbatim}

\begin{semiverbatim}\scriptsize
 \\item<4> \\alert\{Verdad?\}
\end{semiverbatim}

\begin{semiverbatim}\scriptsize
\\end\{itemize\}
\end{semiverbatim}



\column{.45\textwidth}
\begin{itemize}
\item<1-> Lo   \item<3-4> que  \item<3-> empieza
\item<2-3> acaba. \item<4>\alert{Verdad?}
\end{itemize}


\end{columns}
\end{block}






\end{frame}



\begin{frame}

\begin{block}{}

\begin{columns}
\column{.45\textwidth}
\begin{semiverbatim}\scriptsize
\\begin\{itemize\}
\end{semiverbatim}

\begin{semiverbatim}\scriptsize
 \\item<+-| alert@+> Lo
\end{semiverbatim}

\begin{semiverbatim}\scriptsize
 \\item<+-| alert@+> que
\end{semiverbatim}

\begin{semiverbatim}\scriptsize
 \\item<+-| alert@+> empieza
\end{semiverbatim}

\begin{semiverbatim}\scriptsize
 \\item<+-| alert@+> acaba.
\end{semiverbatim}

\begin{semiverbatim}\scriptsize
\\end\{itemize\}
\end{semiverbatim}



\column{.45\textwidth}
\begin{itemize}
 \item<+-| alert@+> Lo   \item<+-| alert@+> que  \item<+-| alert@+> empieza
\item<+-| alert@+> acaba.
\end{itemize}


\end{columns}
\end{block}






\end{frame}


\begin{frame}

\begin{block}{}

\begin{columns}
\column{.45\textwidth}
\begin{semiverbatim}\scriptsize
\\begin\{itemize\}[<+-| alert@+>]
\end{semiverbatim}

\begin{semiverbatim}\scriptsize
 \\item Lo
\end{semiverbatim}

\begin{semiverbatim}\scriptsize
 \\item que
\end{semiverbatim}

\begin{semiverbatim}\scriptsize
 \\item empieza
\end{semiverbatim}

\begin{semiverbatim}\scriptsize
 \\item acaba.
\end{semiverbatim}

\begin{semiverbatim}\scriptsize
\\end\{itemize\}
\end{semiverbatim}



\column{.45\textwidth}
\begin{itemize}
 \item<+-| alert@+> Lo   \item<+-| alert@+> que  \item<+-| alert@+> empieza
\item<+-| alert@+> acaba.
\end{itemize}


\end{columns}
\end{block}






\end{frame}


\begin{frame}

\begin{block}{}

\begin{columns}
\column{.45\textwidth}
\begin{semiverbatim}\scriptsize
\\begin\{itemize\}
\end{semiverbatim}

\begin{semiverbatim}\scriptsize
 \\item<1-|alert@2> Lo
\end{semiverbatim}

\begin{semiverbatim}\scriptsize
 \\item<3-|alert@1> que
\end{semiverbatim}

\begin{semiverbatim}\scriptsize
 \\item<1-2|alert@2> empieza
\end{semiverbatim}

\begin{semiverbatim}\scriptsize
 \\item<2-4> acaba.
\end{semiverbatim}

\begin{semiverbatim}\scriptsize
 \\item<4> \\alert\{Tiene esto fin?\}
\end{semiverbatim}

\begin{semiverbatim}\scriptsize
\\end\{itemize\}
\end{semiverbatim}



\column{.45\textwidth}
\begin{itemize}
\item<1-|alert@2> Lo   \item<3-|alert@1> que
\item<1-2|alert@2> empieza
\item<2-4> acaba. \item<4>\alert{Tiene esto fin?}
\end{itemize}


\end{columns}
\end{block}






\end{frame}


\begin{frame}






\begin{block}{}

\begin{columns}
\column{.20\textwidth}
\begin{itemize}
\item<2-> \alt<2>{\color{magenta} Lo}{\color{red} acaba}
\item<4-> \alt<4>{\color{blue} que}{\color{green} lo }
\item<6-> \alt<6>{\color{blue} lo }{\color{yellow} que}
\item<8-> \alt<8>{\color{blue} acaba}{\color{magenta} lo}
\end{itemize}

\column{.75\textwidth}
\begin{semiverbatim}\scriptsize
\\begin\{itemize\}
\end{semiverbatim}

\begin{semiverbatim}\scriptsize
\\item<2-> \\alt<2>\{\\color\{magenta\} Lo\}\{\\color\{red\} acaba\}
\end{semiverbatim}

\begin{semiverbatim}\scriptsize
\\item<4-> \\alt<4>\{\\color\{blue\} que\}\{\\color\{green\} lo \}
\end{semiverbatim}

\begin{semiverbatim}\scriptsize
\\item<6-> \\alt<6>\{\\color\{blue\} lo\}\{\\color\{yellow\} que\}
\end{semiverbatim}


\begin{semiverbatim}\scriptsize
\\item<8-> \\alt<8>\{\\color\{blue\} acaba\}\{\\color\{magenta\} lo\}
\end{semiverbatim}

\begin{semiverbatim}\scriptsize
\\end\{itemize\}
\end{semiverbatim}




\end{columns}
\end{block}






\end{frame}

\begin{frame}

\begin{block}{}\scriptsize
\begin{semiverbatim}\\only<1>\{Estoy en 1\}\\only<2>\{Estoy en 2\}\\only<3>\{Estoy en 3\}
\end{semiverbatim}
\only<1>{Estoy en 1}\only<2>{Estoy en 2}\only<3>{Estoy en 3}
\end{block}




\begin{block}<4->{}\scriptsize
\begin{semiverbatim}\\uncover<4>\{Estoy en 4\}
\end{semiverbatim}
\uncover<4>{Estoy en 4}
\end{block}



\begin{block}<5->{}\scriptsize
\begin{semiverbatim}\\invisible<5>\{No estoy en 5\}
\end{semiverbatim}
\invisible<5>{No estuve en 5. Estoy en 6}
\end{block}

\begin{block}<7->{}\scriptsize
\begin{semiverbatim}\\alt<6>\{Estoy en 7\}\{Estoy en 8\}
\end{semiverbatim}
\alt<7>{Estoy en 7}{Estoy en 8}
\end{block}

\only<8>{Alberto, no te duermas}










\end{frame}



\end{document}
