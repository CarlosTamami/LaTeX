% $Header$

\documentclass{beamer}

% This file is a solution template for:

% - Talk at a conference/colloquium.
% - Talk length is about 20min.
% - Style is ornate.


% Copyright 2004 by Till Tantau <tantau@users.sourceforge.net>.
%
% In principle, this file can be redistributed and/or modified under
% the terms of the GNU Public License, version 2.
%
% However, this file is supposed to be a template to be modified
% for your own needs. For this reason, if you use this file as a
% template and not specifically distribute it as part of a another
% package/program, I grant the extra permission to freely copy and
% modify this file as you see fit and even to delete this copyright
% notice. 


\mode<presentation>
{
  \usetheme{Warsaw}
  % or ...

  \setbeamercovered{transparent}
  % or whatever (possibly just delete it)
}


\usepackage[spanish]{babel}
% or whatever

\usepackage[utf8]{inputenc}
%\usepackage[latin1]{inputenc}
% or whatever

\usepackage{times}
\usepackage[T1]{fontenc}
% Or whatever. Note that the encoding and the font should match. If T1
% does not look nice, try deleting the line with the fontenc.


\title[Título corto del trabajo] % (optional, use only with long paper titles)
{Título completo en los Proceedings}

\subtitle
{Incluir sólo si el artículo también tienen un Subtítulo}

\author[Autor Principal, Otro Autor] % (optional, use only with lots of authors)
{P.~Autor\inst{1} \and S.~Otro\inst{2}}
% - Give the names in the same order as the appear in the paper.
% - Use the \inst{?} command only if the authors have different
%   affiliation.

\institute[Universidades de Aquí y de  Allá] % (optional, but mostly needed)
{
  \inst{1}%
  Departmento de Matemáticas\\
  Universidad de Aquí
  \and
  \inst{2}%
  Departmento de  Filosofía\\
  Universidad de Allá}
% - Use the \inst command only if there are several affiliations.
% - Keep it simple, no one is interested in your street address.

\date[Curso de {\LaTeX} 2018] % (optional, should be abbreviation of conference name)
{Conferencia sobre \\ Fabulosas Presentaciones  con Beamer, 2018}
% - Either use conference name or its abbreviation.
% - Not really informative to the audience, more for people (including
%   yourself) who are reading the slides online

\subject{Presentaciones con Beamer}
% This is only inserted into the PDF information catalog. Can be left
% out. 



% If you have a file called "university-logo-filename.xxx", where xxx
% is a graphic format that can be processed by latex or pdflatex,
% resp., then you can add a logo as follows:

% \pgfdeclareimage[height=0.5cm]{university-logo}{university-logo-filename}
% \logo{\pgfuseimage{university-logo}}



% Delete this, if you do not want the table of contents to pop up at
% the beginning of each subsection:
\AtBeginSubsection[]
{
  \begin{frame}<beamer>{Outline}
    \tableofcontents[currentsection,currentsubsection]
  \end{frame}
}


% If you wish to uncover everything in a step-wise fashion, uncomment
% the following command: 

%\beamerdefaultoverlayspecification{<+->}


\begin{document}

\begin{frame}
  \titlepage
\end{frame}

\begin{frame}{Outline}
  \tableofcontents
  % You might wish to add the option [pausesections]
\end{frame}


% Structuring a talk is a difficult task and the following structure
% may not be suitable. Here are some rules that apply for this
% solution: 

% - Exactly two or three sections (other than the summary).
% - At *most* three subsections per section.
% - Talk about 30s to 2min per frame. So there should be between about
%   15 and 30 frames, all told.

% - A conference audience is likely to know very little of what you
%   are going to talk about. So *simplify*!
% - In a 20min talk, getting the main ideas across is hard
%   enough. Leave out details, even if it means being less precise than
%   you think necessary.
% - If you omit details that are vital to the proof/implementation,
%   just say so once. Everybody will be happy with that.

\section{Motivación}

\subsection{De qué trata esto.}

\begin{frame}{Títulos Informativos. Usar Letras Mayúsculas.}{Los subtitulos son opcionales.}
  % - A title should summarize the slide in an understandable fashion
  %   for anyone how does not follow everything on the slide itself.

  \begin{itemize}
  \item
    Usar bastantes  \texttt{itemize}.
  \item
    Y frases o sentencias breves y concisas.
  \end{itemize}
\end{frame}

\begin{frame}{Títulos Informativos.}

  Se pueden crear capas (u overlays\dots)
  \begin{itemize}
  \item usando el comando \texttt{pause} :
    \begin{itemize}
    \item
      Primer item.
      \pause
    \item    
      Segundo item.
    \end{itemize}
  \item
    usando además especificaciones concretas, como sigue:
    \begin{itemize}
    \item<3->
      Primer item.
    \item<4->
      Segundo item.
    \end{itemize}
  \item
    o bien usando el comando general \texttt{uncover} :
    \begin{itemize}
      \uncover<5->{\item
        Primer item.}
      \uncover<6->{\item
        Segundo item.}
    \end{itemize}
  \end{itemize}
\end{frame}


\subsection{Trabajo o conocimientos previos}

\begin{frame}{Títulos Informativos.}
\end{frame}

\begin{frame}{Frases cortas y claras.}
\end{frame}



\section{Nuestros Resultados/Contribuciones}

\subsection{Resultados Principales}

\begin{frame}{Títulos Informativos.}
\end{frame}

\begin{frame}{Frases cortas y claras.}
\end{frame}


\subsection{Ideas Básicas y Demostraciones}


\begin{frame}{Ideas Básicas .}
\end{frame}

\begin{frame}{Hipótesis.}
\end{frame}

\begin{frame}{Demostraciones.}
\end{frame}

\section*{Resumen}

\begin{frame}{Resumen}

  % Keep the summary *very short*.
  \begin{itemize}
  \item
    El \alert{primer mensaje} de su  charla en una o dos líneas.
  \item
    El \alert{segundo mensaje} también escueto, si es posible.
  \item
    Tal vez un \alert{tercer mensaje o idea}, pero no mucho más en una sóla transparencia.
  \end{itemize}
  
  % The following outlook is optional.
  \vskip0pt plus.5fill
  \begin{itemize}
  \item
    Resumiendo
    \begin{itemize}
    \item
      El problema que hemos tratado y resuelto.
    \item
      Ahora algo de lo que aún no hemos conseguido y que será la continuación de nuestro trabajo futuro. 
    \end{itemize}
  \end{itemize}
\end{frame}



% All of the following is optional and typically not needed. 
\appendix
\section<presentation>*{\appendixname}
\subsection<presentation>*{Referencias adicionales.}

\begin{frame}[allowframebreaks]
  \frametitle<presentation>{Referencias adicionales}
    
  \begin{thebibliography}{10}
    
  \beamertemplatebookbibitems
  % Start with overview books.

  \bibitem{Autor1990}
    A.~Autor.
    \newblock {\em Libro de casi todo}.
    \newblock Cierta Editorial, 1990.
 
    
  \beamertemplatearticlebibitems
  % Followed by interesting articles. Keep the list short. 

  \bibitem{Alguien2000}
    S.~Alguien.
    \newblock Sobre esto y aquello.
    \newblock {\em Revista de Esto y Aquello}, 2(1):50--100,
    2000.
  \end{thebibliography}
\end{frame}

\end{document}


