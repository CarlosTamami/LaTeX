%\chapter{Caribe y el Pacífico}

Terminada la Guerra de Sucesión, se le confió el buque Peibo del
Primer Lanfranco, barco en calamitoso estado. Un año después, en 1716,
partió hacia La Habana con la Flota de Galeones, con la misión
habitual de escoltar a los barcos mercantes que viajaban a América y
la especial de limpiar de naves corsarias las aguas de la región, que
habían realizado algunas presas el año anterior \cite{qui01}. Cumplida
la misión, Lezo regresó a Cádiz, donde en 1720 obtuvo el mando de un
nuevo Lanfranco, de sesenta y dos cañones y también genovés, como su
homónimo, conocido asimismo como León Franco y Nuestra Señora del
Pilar.

Con este nuevo navío se integró en una escuadra hispano-francesa al
mando de Jean Nicolas Martinet ---francés al servicio de la Corona
española--- y Bartolomé de Urdizu ---segundo de Martinet y capitán del
único buque real que se unió a los que aportaban los corsarios
franceses---, que partió en diciembre de 1716 a América con el cometido
de limpiar de corsarios y piratas los llamados Mares del Sur, o lo que
es lo mismo, las costas del Perú. La escuadra estaba compuesta por
parte española por cuatro buques de guerra y una fragata y, por parte
francesa, por dos navíos de línea. Tras diversos retrasos, el
grueso de la flota alcanzó El Callao el 27 de septiembre de 1717.
Urdizu y Lezo, sin embargo, tuvieron problemas para doblar el Cabo de
Hornos y se retrasaron; alcanzaron El Callao finalmente en enero de
1720, cuando ya las autoridades del Perú habían devuelto a Europa a
los franceses por las tensiones entre las dos partes.

Las primeras operaciones de los marinos españoles encargados de la
reforma de la flota virreinal fueron contra los dos barcos, el Success
(70) y el Speed Well (70) del corsario inglés John Clipperton, que
logró evitar a la flota virreinal durante algún tiempo, pero tuvo
finalmente que abandonar la zona \cite{qui01}. La flota pasó entonces
a desempeñar labores de vigilancia y patrulla en la región, que
acabaron por minar la salud de Urbizu. La mayor parte de las labores
de patrulla, dada la mala salud de este, recayeron en Lezo.

Agotado Urbizu, lo sustituyó el 16 de febrero de 1723, Lezo, con el
título de general de la Armada de Su Católica Majestad y jefe de la
Escuadra del Mar del Sur, por entonces de escaso tamaño. Además
del Lanfranco de Lezo, la formaban los navíos Conquistador y
Triunfador y la fragata Peregrina.

En mayo de 1725, se casó con una limeña de la alta sociedad, Josefa
Pacheco de Bustos y Solís, veinte años más joven; la boda la presidió
el arzobispo de Lima, fray Diego Morcillo y Rubio de Auñón, que hasta
el año anterior había sido virrey del Perú y había establecido buenas
relaciones con Lezo.

Para reforzar la flota que mandaba, hizo reparar los navíos de línea
con que contaba, desguazó y vendió la Peregrina, de cara recuperación
y mal adaptada a las aguas de la región e hizo construir otros dos
navíos. A principios de 1725 zarpó para combatir el corso y el
contrabando de acuerdo a los bandos promulgados el año anterior por el
nuevo virrey. Tras algunas semanas de patrulla, Lezo se topó con una
escuadra holandesa de cinco barcos, que aventajaban a la suya en
artillería. Sin arredrarse, la acometió; tras una denodada lucha
logró derribar el palo mayor de la capitana y apresarla, y puso en
fuga al resto de buques. Más tarde, atacó y se apoderó de una flota
inglesa de seis barcos de guerra, de los que se quedó tres para la
escuadra virreinal.

Estos éxitos y el crecimiento de la flota disuadieron a los enemigos
y, paradójicamente, llevaron al enfrentamiento entre el virrey,
marqués de Castelfuerte, que deseaba reducir la flota para ahorrar
gastos una vez que la situación parecía controlada, y Lezo, que se
oponía a ello. La relación entre ellos también había empeorado por el
nombramiento nepotista del sobrino del virrey para el cargo de
tesorero de los ingresos por comercio marítimo, que contravenía las
disposiciones y del que Lezo se quejó. Mal avenido con el virrey, que
trató de desacreditarle mediante una inspección ---juicio de
residencia--- de su labor que no encontró falta en el desempeño del
marino, disgustado por el desmantelamiento de la flota ---el virrey
prefirió armar corsarios que invertir en reforzar la flota--- y con
mala salud por la larga estancia en la región y las insalubres
travesías, en septiembre de 1727 escribió al secretario de Marina,
José Patiño para quejarse y solicitar su retiro. Patiño aceptó que
dejase el mando de la escuadra del Perú y le llamó a España, pero no
permitió que abandonase la Armada, consciente de su valía.64 El 13 de
febrero de 1728, le relevó como jefe de la flota virreinal y le ordenó
regresar a la península ibérica, pero Lezo, enfermo, no pudo hacerlo
hasta el año siguiente; el 18 de agosto de 1730 arribó con su familia
a Cádiz. Tras librarse de una epidemia de vómito negro que aquejaba a
la ciudad gracias a haberse inmunizado en América, acudió a Sevilla a
visitar al rey, que ya mostraba signos de desequilibrio mental; la
audiencia real tuvo lugar a finales de septiembre o principios de
octubre.

\section{Matrimonio y Descendencia}

El 5 de mayo de 1725, había contraido matrimonio en Lima con la dama
criolla Josefa Pacheco de Bustos, natural de Locumba (actual Tacna), e
hija de los también criollos José Carlos Pacheco y Benavides, y María
Nicolasa de Bustos y Palacios. El matrimonio tuvo siete hijos: Blas67
Fernando, nacido en Lima y primer marqués de Ovieco (1726); Josefa
Atanasia, nacida también en Lima (1728); Cayetano Tomás; Pedro
Antonio; Agustina Antonia; Eduvigis Antonia, que profesó como su
hermana mayor como agustina recoleta; e Ignacia, que casó con el
marqués de Tabalosos. Los cinco hijos menores nacieron en la península
ibérica y, de ellos, las dos hermanas menores, en El Puerto de Santa
María.
