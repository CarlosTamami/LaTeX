\documentclass{article}
\usepackage[utf8]{inputenc}
\usepackage[spanish]{babel}

\usepackage{tabularx}


\title{Entornos {\tt tabular, tabular*} y {\tt tabularx}}

\begin{document}
	
\maketitle

	A continuación vamos a ver tres ejemplos significativos de tablas generadas con los entornos {\tt tabular, tabular*, tabularx}, respectivamente.
	
\vspace{1cm}

\begin{verbatim}
\begin{tabular}{|l|c|r|}
\hline
foo   & bar    & fubar \\
fubar & toobar & foo \\
\hline
\end{tabular}
\end{verbatim}

\begin{tabular}{|l|c|r|}
  \hline
  foo   & bar    & fubar \\
  fubar & toobar & foo \\
  \hline
\end{tabular}

\vspace{1cm}

\begin{verbatim}
\begin{tabular*}{\textwidth}{@{\extracolsep{\fill}}|l|c|r|}
\hline
foo   & bar    & fubar \\
fubar & toobar & foo \\
\hline
\end{tabular*}
\end{verbatim}

\begin{tabular*}{\textwidth}{@{\extracolsep{\fill}}|l|c|r|}
  \hline
  foo   & bar    & fubar \\
  fubar & toobar & foo \\
  \hline
\end{tabular*}

\vspace{1cm}


Como puede notarse tras estos tres ejemplos, en el entorno {\tt tabular} las columnas por defecto serán tan anchas como haga falta para acomodar el contenido de las mismas.

En la versión {\tt tabular*} \LaTeX{} añadirá espacios suplementarios entre las columnas para hacer que dicha tabla sea tan ancha como se haya especificado (con la opción {\tt textwidth} por ejemplo). Sin embargo, la anchura de alguna de las columnas no cambia; fíjese cómo la primera columna permanece igual que antes, mientras que en la columna central el texto sigue sin aparecer centrado.

Por este motivo, si lo que se desea es producir una tabla con columnas igualmente espaciadas que contengan texto o contenido alíneado a la izquierda, entonces la mejor opción sería usar {tt tabularx}, de manera que las nuevas columnas {tt X} permitirán a \LaTeX{} calcular automáticamente la anchura de dicha columna para hacer que la tabla completa aparezca con la anchura prefijada.

\vspace{1cm}

\begin{verbatim}
\begin{tabularx}{\textwidth}{|X|X|X|}
\hline
foo   & bar    & fubar \\
fubar & toobar & foo \\
\hline
\end{tabularx}
\end{verbatim}

\begin{tabularx}{\textwidth}{|X|X|X|}
	\hline
	foo   & bar    & fubar \\
	fubar & toobar & foo \\
	\hline
\end{tabularx}

\vspace{1.5cm}

No obsante, revise todas estas otras opciones.

\vspace{1cm}

\begin{verbatim}
\begin{tabularx}{\textwidth}{|X|c|X|}
\hline
foo   & bar    & fubar \\
fubar & toobar & foo \\
\hline
\end{tabularx}
\end{verbatim}

\begin{tabularx}{\textwidth}{|X|c|X|}
	\hline
	foo   & bar    & fubar \\
	fubar & toobar & foo \\
	\hline
\end{tabularx}

\vspace{1cm}

\begin{verbatim}
\begin{tabularx}{\textwidth}{|l|X|X|}
\hline
foo   & bar    & fubar \\
fubar & toobar & foo \\
\hline
\end{tabularx}
\end{verbatim}

\begin{tabularx}{\textwidth}{|l|X|X|}
	\hline
	foo   & bar    & fubar \\
	fubar & toobar & foo \\
	\hline
\end{tabularx}

\vspace{1cm}

\begin{verbatim}
\begin{tabularx}{\textwidth}{|X|X|r|}
\hline
foo   & bar    & fubar \\
fubar & toobar & foo \\
\hline
\end{tabularx}
\end{verbatim}

\begin{tabularx}{\textwidth}{|X|X|r|}
	\hline
	foo   & bar    & fubar \\
	fubar & toobar & foo \\
	\hline
\end{tabularx}

\vspace{1cm}

\begin{verbatim}
\begin{tabularx}{\textwidth}{|l|X|r|}
\hline
foo   & bar    & fubar \\
fubar & toobar & foo \\
\hline
\end{tabularx}
\end{verbatim}

\begin{tabularx}{\textwidth}{|l|X|r|}
	\hline
	foo   & bar    & fubar \\
	fubar & toobar & foo \\
	\hline
\end{tabularx}


\end{document}
