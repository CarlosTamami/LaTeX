\documentclass[12pt,
               % twocolumn
              ]{article}

% a4large.sty -- fill an A4 (210mm x 297mm) page
% Note: 1 inch = 25.4 mm = 72.27 pt
%       1 pt   =  3.5 mm (approx)

% vertical page layout -- one inch margin top and bottom
\topmargin        0   mm    % top margin less 1 inch
\headheight       0   mm    % height of box containing the head
\headsep          0   mm    % space between the head and the body of the page
\textheight     240   mm    % the height of text on the page
\footskip         7   mm    % distance from bottom of body to bottom of foot

% horizontal page layout -- one inch margin each side
\oddsidemargin    0   mm    % inner margin less one inch on odd pages
\evensidemargin   0   mm    % inner margin less one inch on even pages
\textwidth      159.2 mm    % normal width of text on page

% paragraph setup
\setlength{\parindent}{0em}
%\setlength{\parindent}{4em} % Tabulación de un párrafo nuevo
\setlength{\parskip}{1em} % Distancia en blanco entre un párrafo y otro.
\setlength{\columnsep}{1em}
%\renewcommand{\baselinestretch}{1.5}

\usepackage[utf8]{inputenc}
\usepackage[spanish]{babel}
\usepackage{enumerate}
\usepackage[usenames,dvipsnames]{color}
\definecolor{webgreen}{rgb}{0, 0.5, 0} % less intense green
\definecolor{webblue}{rgb}{0, 0, 0.5}  % less intense blue
\definecolor{webred}{rgb}{0.5, 0, 0}   % less intense red
\definecolor{dblackcolor}{rgb}{0.0,0.0,0.0}
\definecolor{dbluecolor}{rgb}{.01,.02,0.7}
\definecolor{dredcolor}{rgb}{0.8,0,0}
\definecolor{dgraycolor}{rgb}{0.30,0.3,0.30}
\definecolor{RojoAnayelRey}{rgb}{1,.25,.25}
\usepackage[bookmarks=true,
            bookmarksnumbered=false, % true means bookmarks in 
                                     % left window are numbered                         
            bookmarksopen=false,     % true means only level 1
                                     % are displayed.
            colorlinks=true,
            linkcolor=webred,
            %linkcolor=OliveGreen,
            urlcolor=cyan]{hyperref}
            
\newcommand{\HRule}{\rule{\linewidth}{1mm}}
% descomente la siguiente línea y observe la diferencia
% \renewcommand\labelenumii{\theenumi.\arabic{enumii}. } 

\title
{
\HRule
\begin{flushright}
\Huge
\textbf{La Enumeración}\\[3mm]
\Large
\textbf{con}\\
\Huge
\texttt{enumerate}
\end{flushright}
\HRule 
} 
\author{\large Fco. M. García}

%\date{24/01/2016}

\begin{document}

\maketitle

\tableofcontents % Hace el índice de contenidos.

\setcounter{section}{-1}

\section{Introducción}


  Modificación de las etiquetas en la enumeración ordinaria de
  {\LaTeX} bla bla bla
  
  asdasdksaldksaldkaskdlñkk asdlkas kdaskd asdk asdklaskd ñladk asldka sas
  aksdkasdajd asdlkas dkas dasd alsdkasdk asdias dalskd askdaso dasldk 
  
  asd asd asdas dlaskdlasld aslda sdlkasldk askd asldkaskdaslkd asd0asdasd a
  askdlaskdasl kdalskdo2qdklakd 2pdokñdñkdklk akaskdaskd112dasda wqee we

\section{Detallar con item}

Futuros viajes:

\begin{itemize}
    \item Madrid.
    \item Castilla la Mancha.
    \item Castilla y León.
    \begin{itemize}
         \item Segovia.
         \item Ávila.
    \end{itemize}
\end{itemize}

\renewcommand{\labelitemi}{$-$}
\renewcommand{\labelitemii}{$\cdot$}

Futuros viajes:

\begin{itemize}
    \item[$*$] Madrid.
    \item Castilla la Mancha.
    \item Castilla y León.
    \begin{itemize}
        \item Segovia.
        \item Ávila.
    \end{itemize}
\end{itemize}

\section{Enumeración}

  Facetas en un viaje:
  \begin{enumerate}
  \item Culturales
    \begin{enumerate}
    \item Visita a museos
      \begin{enumerate}
      \item científicos \label{en:cien}
      \item históricos
      \end{enumerate}
    \item Visita a exposiciones
    \end{enumerate}
  \item Sociales
    \begin{enumerate}
    \item Paseos por mercados
    \item Contacto con las gentes (en \ref{en:cien})
    \end{enumerate}
  \item Visita a parques
  \item Visita monumentos al aire libre
  \item Itinerario por los edificios modernistas
  \item Visita a anticuarios
  \end{enumerate}

  Ésta es mi lista de compras para hoy:
  \begin{enumerate}[1(]
  \item Manzanas.
  \item Plátanos.
  \item Pescado fresco.
    \begin{enumerate}[a(]
    \item Emperador.
    \item Gallo.
    \end{enumerate}
  \end{enumerate}

  Haga los siguientes ejercicios:
  \begin{enumerate}[{Ejercicio} 1.]
  \item Visitar tres lugares. \label{en:visitar}
  \item Leer tres libros.
  \item Conocer a tres personas.
  \end{enumerate}  



\section{Comenzar Donde lo Dejamos}

%http://minisconlatex.blogspot.com.es/2011/11/listas-y-enumeraciones.html
%http://minisconlatex.blogspot.com.es/2012/05/como-continuar-con-la-numeracion-entre.html

Leído en \href{https://goo.gl/txW3PR}{minisconlatex}

\newcounter{nx}

Lista de la compra:
\begin{enumerate}
  \setcounter{enumi}{-1}
\item Manzanas.
\item Plátanos.
\item Fresas.
  \setcounter{nx}{\value{enumi}} % le damos al contador el valor de la enumeración.
\end{enumerate}

Continuación de la lista de la compra:

\begin{enumerate}
  \setcounter{enumi}{\value{nx}} % reiniciamos la enumeración
                                 % con el valor del contador.
%  \setcounter{enumi}{14}
\item Limones.
\item Naranjas.
\item Pomelos.
\end{enumerate}

Otra posibilidad de continuación de lista de la compra:

\begin{enumerate}[{2.}i]
  \setcounter{enumi}{\value{nx}} % reiniciamos la enumeración
                                 % con el valor del contador.
\item Limones.
\item Naranjas.
\item Pomelos.
\end{enumerate}

\section{Comenzar Donde lo Dejamos y Cambio de Etiquetas}

\newcounter{mx} % creamos un contador con el nombre "nx".

Primera lista de la compra:

\begin{enumerate}[(A)]
\item Manzanas.
\item Plátanos.
\item Fresas.
\setcounter{mx}{\value{enumi}} % le damos al contador el valor de la enumeración.
\end{enumerate}

Segunda lista de la compra:

\begin{enumerate}[(A)]
  \setcounter{enumi}{\value{mx}} % reiniciamos la enumeración con el valor del contador.
\item Limones.
\item Naranjas.
\item Pomelos.
\end{enumerate}

\section{Descripción}

Fauna típica de los países:
\begin{description}
\item[Australia:] Érase una vez un lugar encantado, no con encanto, que pude visitar y no sé ni como lo imaginé ni como me desplacé hasta esa latitud.
%\item Canguro.
\item[EEUU:] \textit{Águila calva}.
\item[España:] \textbf{Toro}.
\item[México:] \textsc{Águila real}.
\end{description}

  Haga caso a lo que dice \hyperref[en:visitar]{Ejercicio
    \ref*{en:visitar}}.

\end{document}