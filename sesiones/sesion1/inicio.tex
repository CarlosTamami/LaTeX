\documentclass[12pt,
               spanish
              ]{article}
\usepackage[utf8]{inputenc}  % selección de la codificación de los acentos
\usepackage{babel}
\usepackage{amsmath,amsthm}
\usepackage{amsfonts,amssymb,latexsym}
\usepackage{enumerate}

\begin{document}

\section*{Introducción}
\label{sec:cero}

\begin{flushleft}
En un lugar de la Mancha, de cuyo nombre no quiero acordarme, no ha
mucho tiempo que vivía un hidalgo de los de lanza en astillero, adarga
antigua, rocín flaco y galgo corredor. Una olla de algo más vaca que
carnero, salpicón las más noches, duelos y quebrantos los sábados,
lentejas los viernes, algún palomino de añadidura los domingos,
consumían las tres partes de su hacienda. El resto della concluían
sayo de velarte, calzas de velludo para las fiestas con sus pantuflos
de lo mismo, los días de entre semana se honraba con su vellori de lo
más fino. Tenía en su casa una ama que pasaba de los cuarenta, y una
sobrina que no llegaba a los veinte, y un mozo de campo y plaza, que
así ensillaba el rocín como tomaba la podadera. Frisaba la edad de
nuestro hidalgo con los cincuenta años, era de complexión recia, seco
de carnes, enjuto de rostro; gran madrugador y amigo de la
caza. Quieren decir que tenía el sobrenombre de Quijada o Quesada (que
en esto hay alguna diferencia en los autores que deste caso escriben),
aunque por conjeturas verosímiles se deja entender que se llama
Quijana; pero esto importa poco a nuestro cuento; basta que en la
narración dél no se salga un punto de la verdad.  
\end{flushleft}

\begin{flushleft}
Es, pues, de saber, que este sobredicho hidalgo, los ratos que estaba
ocioso (que eran los más del año) se daba a leer libros de caballerías
con tanta afición y gusto, que olvidó casi de todo punto el ejercicio
de la caza, y aun la administración de su hacienda; y llegó a tanto su
curiosidad y desatino en esto, que vendió muchas hanegas de tierra de
sembradura, para comprar libros de caballerías en que leer; y así
llevó a su casa todos cuantos pudo haber dellos; y de todos ningunos
le parecían tan bien como los que compuso el famoso Feliciano de
Silva: porque la claridad de su prosa, y aquellas intrincadas razones
suyas, le parecían de perlas; y más cuando llegaba a leer aquellos
requiebros y cartas de desafío, donde en muchas partes hallaba
escrito: la razón de la sinrazón que a mi razón se hace, de tal manera
mi razón enflaquece, que con razón me quejo de la vuestra fermosura, y
también cuando leía: los altos cielos que de vuestra divinidad
divinamente con las estrellas se fortifican, y os hacen merecedora del
merecimiento que merece la vuestra grandeza. Con estas y semejantes
razones perdía el pobre caballero el juicio, y desvelábase por
entenderlas, y desentrañarles el sentido, que no se lo sacara, ni las
entendiera el mismo Aristóteles, si resucitara para sólo ello. No
estaba muy bien con las heridas que don Belianis daba y recibía,
porque se imaginaba que por grandes maestros que le hubiesen curado,
no dejaría de tener el rostro y todo el cuerpo lleno de cicatrices y
señales; pero con todo alababa en su autor aquel acabar su libro con
la promesa de aquella inacabable aventura, y muchas veces le vino
deseo de tomar la pluma, y darle fin al pie de la letra como allí se
promete; y sin duda alguna lo hiciera, y aun saliera con ello, si
otros mayores y continuos pensamientos no se lo estorbaran.
\end{flushleft}

\section{Primera Sección}
\label{primeraSeccion}
¡Hola ... mundo! y ya \ldots está

Estoy escribiendo mi primer fichero LaTeX pero me gusta más así
\LaTeX.

Enumeraré mis intenciones:

\begin{enumerate}
\item Encontrar mejoras.
\item Implementarlas.
\item Explicar la estructura básica de la cabecera.\label{primera}
\end{enumerate}

Para la \ref{primera}, ya estoy en ello.

\begin{equation}
  \label{eq:primera}
  \sum_{i=1}^{n}\frac{1}{i},\qquad\text{ donde } n \text{ es natural }
\end{equation}
como decíamos en la ecuación (\ref{eq:primera})

\section{Segunda sección}
\label{sec:segunda}

Como decíamos en la sección \ref{sec:cero}
\end{document}