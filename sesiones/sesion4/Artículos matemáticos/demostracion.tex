%%% Local Variables: 
%%% mode: latex
%%% TeX-master: "bbp_formula"
%%% End:

\section{Demostración de la Fórmula BBP}

\begin{lemma}
  Para todo número natural $n$ se cumple:
  \begin{equation*}
    \sum_{k=0}^{\infty}\frac{1}{16^{k}(8k+n)}
    =\sqrt{2}^{n}\int_{0}^{\frac{1}{\sqrt{2}}}
    \left(\frac{x^{n-1}}{1-x^{8}}\right)
    \,dx
  \end{equation*}
\end{lemma}

\begin{proof}
  Sea $n$ un número natural cualquiera. Entonces:
  \begin{align*}
    \sum_{k=0}^{\infty}\frac{1}{16^{k}(8k+n)}
         &=\sqrt{2}^{n}\sum_{k=0}^{\infty}
             \frac{\left(\frac{1}{\sqrt{2}}\right)^{8k+n}}{8k+n}\\
         &=\sqrt{2}^{n}\sum_{k=0}^{\infty}
             \left[\frac{x^{8k+n}}{8k+n}\right]_{0}^{\frac{1}{\sqrt{2}}}\\
         &=\sqrt{2}^{n}\sum_{k=0}^{\infty}
           \left(
             \int _{0}^{\frac{1}{\sqrt{2}}}x^{8k+n-1}\,dx
           \right)\\
         &=\sqrt{2}^{n}\int _{0}^{\frac{1}{\sqrt{2}}}
           \left(\sum_{k=0}^{\infty}
                  x^{8k+n-1}
           \right)\,dx\\
         &=\sqrt{2}^{n}\int _{0}^{\frac{1}{\sqrt{2}}}
           \left(x^{n-1}\sum_{k=0}^{\infty}
                  (x^{8})^{k}
           \right)\,dx\\
             &=\sqrt{2}^{n}\int _{0}^{\frac{1}{\sqrt{2}}}
           \left(x^{n-1}\frac{1}{1-x^{8}}
           \right)\,dx\\
  \end{align*}
  
\end{proof}

\begin{theorem}
  Es cierta la siguiente igualdad (denominada
  \hyperref[eq:bbp]{fórmula BBP}):
  \begin{equation}
    \label{eq:bbp}
        \pi = \sum_{k=0}^{\infty}\frac{1}{16^{k}}
    \left(
      \frac{4}{8k+1}
      -\frac{2}{8k+4}
      -\frac{1}{8k+5}
      -\frac{1}{8k+6}
    \right) \nonumber
  \end{equation}
\end{theorem}

\begin{proof}
  Son ciertas las siguientes igualdades:
  \begin{align*}
    &\sum_{k=0}^{\infty}\frac{1}{16^{k}}
    \left(
      \frac{4}{8k+1}
      -\frac{2}{8k+4}
      -\frac{1}{8k+5}
      -\frac{1}{8k+6}
    \right)\\
     &=
     4\sum_{k=0}^{\infty}\frac{1}{16^{k}(8k+1)}
    -2\sum_{k=0}^{\infty}\frac{1}{16^{k}(8k+4)}
     -\sum_{k=0}^{\infty}\frac{1}{16^{k}(8k+5)}
     -\sum_{k=0}^{\infty}\frac{1}{16^{k}(8k+6)}\\
     &=4\left(\sqrt{2}^{1}\int_{0}^{\frac{1}{\sqrt{2}}}\frac{x^{1-1}}{1-x^{8}}\,dx\right)
     -2\left(\sqrt{2}^{4}\int_{0}^{\frac{1}{\sqrt{2}}}\frac{x^{4-1}}{1-x^{8}}\,dx\right)
     -\left(\sqrt{2}^{5}\int_{0}^{\frac{1}{\sqrt{2}}}\frac{x^{5-1}}{1-x^{8}}\,dx\right)\\
     &-\left(\sqrt{2}^{6}\int_{0}^{\frac{1}{\sqrt{2}}}\frac{x^{6-1}}{1-x^{8}}\,dx\right)\\
     &=\int_{0}^{\frac{1}{\sqrt{2}}}\frac{4\sqrt{2}-8x^{3}-4\sqrt{2}x^{4}-8x^{5}}
                                     {1-x^{8}}\,dx\\
  \end{align*}
  Haremos ahora el cambio de variable:
  \begin{equation*}
    \begin{cases}
      y=x\sqrt{2}&\\
      dy=dx\sqrt{2}&
    \end{cases}
  \end{equation*}
  Sustituyendo resulta lo siguiente:
  \begin{align*}
    \int_{0}^{\frac{1}{\sqrt{2}}}\frac{4\sqrt{2}-8x^{3}-4\sqrt{2}x^{4}-8x^{5}}
                                     {1-x^{8}}\,dx
    &=\bigint\limits_{0}^{1}\left(
      \frac{4\sqrt{2}-\frac{8}{\sqrt{2}^{3}}y^{3}
      -\frac{4\sqrt{2}}{\sqrt{2}^{4}}y^{4}
      -\frac{8}{\sqrt{2}^{5}}y^{5}}
       {1-\frac{1}{\sqrt{2}^{8}}y^{8}}\right)\,\frac{dy}{\sqrt{2}}\\
    &=16\int_{0}^{1}\frac{4-2y^{3}-y^{4}-y^{5}}{16-y^{8}}\,dy\\
    &=16\int_{0}^{1}\frac{y-1}{(y^{2}-2y+2)(y^{2}-2)}\,dy\\
  \end{align*}
  Descomponiendo en fracciones simples:
  \begin{align*}
    &=\int_{0}^{1}\left(
       \frac{8-4y}{y^{2}-2y+2}+\frac{4y}{y^{2}-2}\right)\,dy\\[3mm]
    &=-2\int_{0}^{1}\frac{2y-2}{y^{2}-2y+2}\,dy
      +4\int_{0}^{1}\frac{1}{1+(y-1)^{2}}\,dy
      +2\int_{0}^{1}\frac{2y}{y^{2}-2}\,dy\\[3mm]
    &=-2[\ln(y^{2}-2y+2)]_{0}^{1}
     +4[\arctan(y-1)]_{0}^{1}
     +2[\ln(2-y^{2})]_{0}^{1}\\[3mm]
    &=-2\ln(1)+2\ln(1)+4\arctan(0)-4\arctan(-1)+2\ln(1)-2\ln(2)\\[3mm]
    &=4\arctan(1)\\[3mm]
    &=\pi
  \end{align*}
\end{proof}