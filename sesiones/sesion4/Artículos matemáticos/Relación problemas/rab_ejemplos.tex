\section{Ejemplos}

\begin{exercise}
  \label{ex:leqEquivalent}
  Sea $\mathbf{B}=\langle B,+,\cdot,\bar{\ },0,1\rangle$ un álgebra de
  Boole. Demuestre que para todo $x,y\in B$ son equivalentes las
  siguientes afirmaciones:
  \begin{enumerate}
  \item $x\leq y$
  \item $x+y=y$
  \item $x\to y=1$
  \end{enumerate}
\end{exercise}

\begin{solution}
  Sean $x,y\in B$ cualesquiera y supongamos que $x\leq y$, es decir,
  que $xy=x$. Entonces:
  \begin{align*}
    x+y&=xy+y\\
       &=y+yx\\
       &=y
  \end{align*}
Supongamos ahora que $x+y=y$; entonces:
\begin{align*}
  x\to y&=x\to(x+y)\\
        &=x\to(x'\to y)\\
        &=x'+x+y\\
        &=1+y\\
        &=1
\end{align*}
Finalmente, supongamos que $x\to y=1$; entonces:
\begin{align*}
  x&=x1\\
   &=x(x\to y)\\
   &=x(x'+y)\\
   &=xx'+xy\\
   &=0+xy\\
   &=xy
\end{align*}
\end{solution}

\begin{remark}
  En el ambiente del \hyperref[ex:leqEquivalent]{Ejercicio
    \ref*{ex:leqEquivalent}}, supongamos que $xy=1$; entonces:
  \begin{align*}
    x\to y&=xy\to y\\
          &=x\to(y\to y)\\
          &=x\to 1\\
          &=1
  \end{align*}
\end{remark}

\begin{exercise}
  \label{ex:nandConsequence}
  Sea $\mathbf{B}=\langle B,+,\cdot,\bar{\ },0,1\rangle$ un álgebra de
  Boole. Demuestre que para todo $x,y\in B$, si $x\uparrow y=0$
  entonces $x=1=y$ 
\end{exercise}

\begin{proof}
  Sean $x,y\in B$ tales que $x\uparrow y=0$, o sea, $0=(xy)'$. Esto es
  equivalente a que $xy=1$, es decir, $x=1=y$.
\end{proof}

\begin{exercise}
  \label{ex:equiv}
  Sea $\mathbf{B}$ un álgebra de Boole
  \textbf{cualquiera}. Demuestre que para todo $a,b,c\in B$ son
  equivalentes las siguientes afirmaciones:
  \begin{enumerate}
  \item $a\oplus b\leq c$
  \item $ac'=bc'$
  \end{enumerate}
\end{exercise}

% \begin{solution}
%   Para cualesquiera $a,b,c\in B$,
%   \begin{align*}
%     ac'(bc')'&=ac'(b'+c'')\\
%              &=ac'(b'+c)\\
%              &=ac'b'+ac'c\\
%              &=ab'c'+a0\\
%              &=ab'c'+0\\
%              &=ab'c'
%   \end{align*}
%   Por tanto:
%   \begin{equation}
%     \label{eq:par18192A}
%     ac'(bc')'=ab'c'
%   \end{equation}
%   y también (como ecuación es idéntica):
%   \begin{equation}
%     \label{eq:par18192B}
%     bc'(ac')'=a'bc'
%   \end{equation}
%   Supongamos ahora que $a\oplus b\leq c$, o equivaléntemente que
%   $ab'+a'b\leq c$. Entonces, por transitividad, $ab'\leq c$ y
%   $a'b\leq c$ o lo que es equivalente: $ab'c'=0$ y
%   $a'bc'=0$. Entonces, por
%   \hyperref[eq:par18192A]{(\ref*{eq:par18192A})}, $ac'(bc')'=0$ y por
%   \hyperref[eq:par18192B]{(\ref*{eq:par18192B})}, $bc'(ac')'=0$. De lo
%   primero se deduce que $ac'\leq bc'$ y de lo segundo que
%   $bc'\leq ac'$. Por la antisimetría de $\leq$, se tiene que
%   $ac'=bc'$. \textbf{Recíprocamente}, supongamos que
%   $ac'=bc'$. Entonces $ac'(bc')'=0$ y $bc'(ac')'=0$; como antes
%   $ab'c'=0=a'bc'$. Deducimos de esto que $ab'\leq c$ y que $a'b\leq
%   c$, de donde:
%   \begin{equation*}
%     a\oplus b = ab'+a'b\leq c
%   \end{equation*}
% \end{solution}

\begin{exercise}
  Sea $n$ un número natural y sean
  $f,g,k,h\colon B_{2}^{n}\longrightarrow B_{2}$. Considere la
  función:
  \begin{equation*}
    h(x_{0},\ldots,x_{n-1})=
    \begin{cases}
      f(x_{0},\ldots,x_{n-1}),&\text{ si } k(x_{0},\ldots,x_{n-1})=1\\
      g(x_{0},\ldots,x_{n-1}),&\text{ si } k(x_{0},\ldots,x_{n-1})=0
    \end{cases}
  \end{equation*}
  y demuestre que 
  \begin{equation}
    \label{eq:2}
    h=kf+\bar{k}g
  \end{equation}
  Seguidamente considere el caso
  particular de $n=4$, $f,g,h\colon B_{2}^{4}\longrightarrow B_{2}$
  definidas por: $f(x,y,z,t)=x\uparrow z$, $f(x,y,z,t)=\bar{z}$ y $h$
  según lo siguiente
  \begin{equation*}
    h(x,y,z,t)=
    \begin{cases}
      f(x,y,z),& \text{ si }\bar{x}\leq\bar{t}\\
      g(y,z,t),& \text{ si }x<t 
    \end{cases}
  \end{equation*}
  Aplique \hyperref[eq:2]{(\ref*{eq:2})} a este caso para obtener:
  \begin{equation}
    \label{eq:3}
    h(x,y,z,t)=(t\to x)(x\uparrow z)+\bar{z}
  \end{equation}
  \end{exercise}

  \begin{solution}
    Supongamos que para
    $\langle x_{0},\ldots,x_{n-1}\rangle\in B_{2}^{n}$, se tiene
    $k(x_{0},\ldots,x_{n-1})=1$; entonces
    \begin{align*}
      h(x_{0},\ldots,x_{n-1})&=k(x_{0},\ldots,x_{n-1})f(x_{0},\ldots,x_{n-1})
                            +\bar{k}(x_{0},\ldots,x_{n-1})g(x_{0},\ldots,x_{n-1})\\
                           &=1f(x_{0},\ldots,x_{n-1})
                            +0g(x_{0},\ldots,x_{n-1})\\
                           &=f(x_{0},\ldots,x_{n-1})
    \end{align*}
    y si $k(x_{0},\ldots,x_{n-1})=0$ se tiene análoga e evidentemente
    que $h(x_{0},\ldots,x_{n-1})=g(x_{0},\ldots,x_{n-1})$. Por otra
    parte, $\bar{x}\leq\bar{t}$ es equivalente a $t\leq x$ y, según el
    \hyperref[ex:leqEquivalent]{Ejercicio \ref*{ex:leqEquivalent}},
    esto equivalente a $t\to x=1$; por tanto, aquí
    $k(x,y,z,t)=t\to x$. La particularización de
    \hyperref[eq:2]{(\ref*{eq:2})} es:
    \begin{equation*}
      (t\to x)(x\uparrow z)+\bar{z}
    \end{equation*}
    pues: si $t\to x=0$ entonces
    $(t\to x)(x\uparrow z)+\bar{z}=\bar{z}$ y si $t\to x=1$, caben dos
    casos:
    \begin{itemize}
    \item $x\uparrow z=1$; entonces 
      \begin{align*}
        (t\to x)(x\uparrow z)+\bar{z}&=1+\bar{z}\\
                       &=1\\
                       &=x\uparrow z
      \end{align*}
    \item $x\uparrow z=0$; entonces
      (cfr. \hyperref[ex:nandConsequence]{Ejercicio
        \ref*{ex:nandConsequence}}) $\bar{z}=0$, luego:
      \begin{align*}
        (t\to x)(x\uparrow z)+\bar{z}&=0+\bar{z}\\
                       &=0+0\\
                       &=0\\
                       &=x\uparrow z
      \end{align*}
    \end{itemize}
    de lo que deducimos que si $t\to x=1$ entonces
    $(t\to x)(x\uparrow z)+\bar{z}=x\uparrow z$.
  \end{solution}

\begin{example}
  \label{ex:digitos7segmentos}
  % http://www.zeepedia.com/read.php?converting_between_pos_and_sop_using_the_k-map_digital_logic_design&b=9&c=11
  El dígito a 7 segmentos se forma iluminando en la pantalla los
  segmentos apropiados según explica esquemáticamente la
  \hyperref[fg:digitos7segmentos]{Figura \ref*{fg:digitos7segmentos}}
  y así puede mostrar los números decimales del 0 al 9. El dígito
  queda construido por iluminación de hasta los segmentos: a,b,c,d,e,f
  y g, que se activan o desactivan mediante un circuito digital
  dependiendo del número que ha de ser mostrado; por ejemplo, el
  dígito 3 requiera la activación de exáctamente los segmentos:
  a,b,c,d y g. Para mostrar el dígito 7 quedarán desactivados
  exactamente los segmentos: d, e, f y g.

  El circuito que activa los segmentos adecuados para mostrar uno
  cualquiera de los dígitos se conoce como \textit{Decodificador BCD a
    7 segmentos};\index{decodificador BCD a
    7 segmentos} su entrada es un número BCD de 4-bit entre 0 y
  9. Cada una de las 7 salidas del circuito conectan con los siete
  segmentos.

  Para implementar el circuito decodificador de cuatro entradas
  ($n=4$) y siete salidas ($m=7$) debemos hacer una tabla para cada
  segmento, representando su estado para combinación de entradas. Así
  pues debemos determinar siete expresiones, una por cada segmento,
  antes de implementar el circuito.

  Puesto que los números representados por una entrada de 4 bits son
  16, la tabla de cada función tendrá 16 entradas combinacionales. No
  obstante las seis últimas combinaciones son ``no-importa'', ya que
  ninguna de ellas ocurre al ser las entradas de números BCD de 4
  bits. Los estados no-importa ayudarán sin embargo a simplificar las
  expresiones booleanas para los siete segmentos. Según lo que
  explican la: \hyperref[fg:segmentosAaF]{Figura
    \ref*{fg:segmentosAaF}}, \hyperref[fg:KsegmentosAaF]{Figura
    \ref*{fg:KsegmentosAaF}} y
  \hyperref[fg:segmentoG]{\ref*{fg:segmentoG}}, la función booleana
  $D_{BCD7}(a,b,c,d)$ que codifica el \textit{Decodificador BCD a
    7 segmentos} es la siguiente:
  \begin{align*}
    D_{BCD7}(a,b,c,d)=&\langle a+c+bd+\bar{b}\bar{d},\\
                    &\bar{b}+\bar{c}\bar{d}+cd,\\
                    &b+\bar{c}+d,\\
                    &a+\bar{b}\bar{d}+\bar{b}c+c\bar{d}+b\bar{c}d,\\
                    &\bar{b}\bar{d}+c\bar{d},\\
                    &a+b\bar{c}+b\bar{d}+\bar{c}\bar{d},\\
                    &a+b\bar{c}+\bar{b}c+c\bar{d}\rangle
  \end{align*}
  donde las entradas de la 7-upla que representa a $D_{BCD7}(a,b,c,d)$
  corresponden a los segmentos en el orden de la plabra abcdefg.
\end{example}

\begin{figure}[!hbp]
  \centering
  \mbox{
    \subfigure[dígito a 7 segmentos]{
      \label{sfg:digitos}
      \includegraphics[width=.20\textwidth]{digitos.png}
    }
    \qquad
    \subfigure[segmentos activos por cada dígito]{
      \label{sfg:digitosSegmentos}
      \begin{tabular}[b]{|c|l|}
        \hline
        Dígito&\multicolumn{1}{|c|}{Segmentos}\\\hline
        0&a,b,c,d,e,f\\
        1&b,c\\
        2&a,b,d,e,g\\
        3&a,b,c,d,g\\
        4&b,c,f,g\\
        5&a,c,d,f,g\\
        6&a,c,d,e,f,g\\
        7&a,b,c\\
        8&a,b,c,d,e,f,g\\
        9&a,b,c,d,f,g\\\hline
      \end{tabular}
    }
  }
  \caption{\label{fg:digitos7segmentos} Dígitos a 7 segmentos y su
    formación activándolos.}
\end{figure}

\begin{figure}[!hbp]
  \centering
  \mbox{
    \subfigure[Codif. segmento a]{
      \label{sfg:segmentoA}
      \begin{tabular}[b]{|c|c|c|c|c|}
        \hline
        \multicolumn{4}{|c|}{Input}&\multicolumn{1}{|c|}{Output}\\\hline
        a&b&c&d&segmento a\\\hline
        0&0&0&0&1\\\hline
        0&0&0&1&0\\\hline
        0&0&1&0&1\\\hline
        0&0&1&1&1\\\hline
        0&1&0&0&0\\\hline
        0&1&0&1&1\\\hline
        0&1&1&0&1\\\hline
        0&1&1&1&1\\\hline
        1&0&0&0&1\\\hline
        1&0&0&1&1\\\hline
        1&0&1&0&x\\\hline
        1&0&1&1&x\\\hline
        1&1&0&0&x\\\hline
        1&1&0&1&x\\\hline
        1&1&1&0&x\\\hline
        1&1&1&1&x\\\hline
      \end{tabular}
    }

    \subfigure[Codif. segmento b]{
      \label{sfg:segmentoB}
      \begin{tabular}[b]{|c|c|c|c|c|}
        \hline
        \multicolumn{4}{|c|}{Input}&\multicolumn{1}{|c|}{Output}\\\hline
        a&b&c&d&segmento b\\\hline
        0&0&0&0&1\\\hline
        0&0&0&1&1\\\hline
        0&0&1&0&1\\\hline
        0&0&1&1&1\\\hline
        0&1&0&0&1\\\hline
        0&1&0&1&0\\\hline
        0&1&1&0&0\\\hline
        0&1&1&1&1\\\hline
        1&0&0&0&1\\\hline
        1&0&0&1&1\\\hline
        1&0&1&0&x\\\hline
        1&0&1&1&x\\\hline
        1&1&0&0&x\\\hline
        1&1&0&1&x\\\hline
        1&1&1&0&x\\\hline
        1&1&1&1&x\\\hline
      \end{tabular}
    }
    \subfigure[Codif. segmento c]{
      \label{sfg:segmentoC}
      \begin{tabular}[b]{|c|c|c|c|c|}
        \hline
        \multicolumn{4}{|c|}{Input}&\multicolumn{1}{|c|}{Output}\\\hline
        a&b&c&d&segmento c\\\hline
        0&0&0&0&1\\\hline
        0&0&0&1&1\\\hline
        0&0&1&0&0\\\hline
        0&0&1&1&1\\\hline
        0&1&0&0&1\\\hline
        0&1&0&1&1\\\hline
        0&1&1&0&1\\\hline
        0&1&1&1&1\\\hline
        1&0&0&0&1\\\hline
        1&0&0&1&1\\\hline
        1&0&1&0&x\\\hline
        1&0&1&1&x\\\hline
        1&1&0&0&x\\\hline
        1&1&0&1&x\\\hline
        1&1&1&0&x\\\hline
        1&1&1&1&x\\\hline
      \end{tabular}
    }
  }
  \mbox{
        \subfigure[Codif. segmento d]{
      \label{sfg:segmentoD}
      \begin{tabular}[b]{|c|c|c|c|c|}
        \hline
        \multicolumn{4}{|c|}{Input}&\multicolumn{1}{|c|}{Output}\\\hline
        a&b&c&d&segmento d\\\hline
        0&0&0&0&1\\\hline
        0&0&0&1&0\\\hline
        0&0&1&0&1\\\hline
        0&0&1&1&1\\\hline
        0&1&0&0&0\\\hline
        0&1&0&1&1\\\hline
        0&1&1&0&1\\\hline
        0&1&1&1&0\\\hline
        1&0&0&0&1\\\hline
        1&0&0&1&1\\\hline
        1&0&1&0&x\\\hline
        1&0&1&1&x\\\hline
        1&1&0&0&x\\\hline
        1&1&0&1&x\\\hline
        1&1&1&0&x\\\hline
        1&1&1&1&x\\\hline
      \end{tabular}
    }
    
    \subfigure[Codif. segmento e]{
      \label{sfg:segmentoE}
      \begin{tabular}[b]{|c|c|c|c|c|}
        \hline
        \multicolumn{4}{|c|}{Input}&\multicolumn{1}{|c|}{Output}\\\hline
        a&b&c&d&segmento e\\\hline
        0&0&0&0&1\\\hline
        0&0&0&1&0\\\hline
        0&0&1&0&1\\\hline
        0&0&1&1&0\\\hline
        0&1&0&0&0\\\hline
        0&1&0&1&0\\\hline
        0&1&1&0&1\\\hline
        0&1&1&1&0\\\hline
        1&0&0&0&1\\\hline
        1&0&0&1&0\\\hline
        1&0&1&0&x\\\hline
        1&0&1&1&x\\\hline
        1&1&0&0&x\\\hline
        1&1&0&1&x\\\hline
        1&1&1&0&x\\\hline
        1&1&1&1&x\\\hline
      \end{tabular}
    }

    \subfigure[Codif. segmento f]{
      \label{sfg:segmentoF}
      \begin{tabular}[b]{|c|c|c|c|c|}
        \hline
        \multicolumn{4}{|c|}{Input}&\multicolumn{1}{|c|}{Output}\\\hline
        a&b&c&d&segmento f\\\hline
        0&0&0&0&1\\\hline
        0&0&0&1&0\\\hline
        0&0&1&0&0\\\hline
        0&0&1&1&0\\\hline
        0&1&0&0&1\\\hline
        0&1&0&1&1\\\hline
        0&1&1&0&1\\\hline
        0&1&1&1&0\\\hline
        1&0&0&0&1\\\hline
        1&0&0&1&1\\\hline
        1&0&1&0&x\\\hline
        1&0&1&1&x\\\hline
        1&1&0&0&x\\\hline
        1&1&0&1&x\\\hline
        1&1&1&0&x\\\hline
        1&1&1&1&x\\\hline
      \end{tabular}
    }
  }
  \caption{\label{fg:segmentosAaF} Codificación por segmentos de su activación del a al f.}
\end{figure}

\begin{figure}[!hbp]
  \centering
  \mbox{
    \subfigure[Mapa K segmento a: $a+c+bd+\bar{b}\bar{d}$]{
      \label{sfg:KsegmentoA}
      \begin{karnaugh-map}[4][4][1][$cd$][$ab$]
        \manualterms{1,0,1,1,0,1,1,1,1,1,x,x,x,x,x,x}
        \implicantcorner[0,2]
        \implicant{12}{10}
        \implicant{5}{15}
        \implicant{3}{10}
      \end{karnaugh-map}
    }
    \subfigure[Mapa K segmento b: $\bar{b}+\bar{c}\bar{d}+cd$]{
      \label{sfg:KsegmentoB}
      \begin{karnaugh-map}[4][4][1][$cd$][$ab$]
        \manualterms{1,1,1,1,1,0,0,1,1,1,x,x,x,x,x,x}
        \implicantedge{0}{2}{8}{10}
        \implicant{0}{8}
        \implicant{3}{11}
      \end{karnaugh-map}
    }
  }
  \mbox{    
    \subfigure[Mapa K segmento c: $b+\bar{c}+d$]{
      \label{sfg:KsegmentoC}
      \begin{karnaugh-map}[4][4][1][$cd$][$ab$]
        \manualterms{1,1,0,1,1,1,1,1,1,1,x,x,x,x,x,x}
        \implicant{0}{9}
        \implicant{1}{11}
        \implicant{4}{14}
      \end{karnaugh-map}
    }

    \subfigure[Mapa K segmento d: $a+\bar{b}\bar{d}+\bar{b}c+c\bar{d}+b\bar{c}d$]{
      \label{sfg:KsegmentoD}
      \begin{karnaugh-map}[4][4][1][$cd$][$ab$]
        \manualterms{1,0,1,1,0,1,1,0,1,1,x,x,x,x,x,x}
        \implicantcorner[0,2]
        \implicant{12}{10}
        \implicantedge{3}{2}{11}{10}
        \implicant{2}{10}
        \implicant{5}{13}
      \end{karnaugh-map}   
    }
  }
  \mbox{
    \subfigure[Mapa K segmento e: $\bar{b}\bar{d}+c\bar{d}$]{
      \label{sfg:KsegmentoE}
      \begin{karnaugh-map}[4][4][1][$cd$][$ab$]
        \manualterms{1,0,1,0,0,0,1,0,1,0,x,x,x,x,x,x}
        \implicantcorner[0,2]
        \implicant{2}{10}
      \end{karnaugh-map}
    }
    \subfigure[Mapa K segmento f: $a+b\bar{c}+b\bar{d}+\bar{c}\bar{d}$]{
      \label{sfg:KsegmentoF}
      \begin{karnaugh-map}[4][4][1][$cd$][$ab$]
        \manualterms{1,0,0,0,1,1,1,0,1,1,x,x,x,x,x,x}
        \implicant{0}{8}
        \implicant{4}{13}
        \implicant{12}{10}
        \implicantedge{4}{12}{6}{14}
      \end{karnaugh-map}
    }
  }
  \caption{\label{fg:KsegmentosAaF} Mapas K para las funciones de
    codificación de la activación de los segmentos a a f}
\end{figure}

\begin{figure}[!hbp]
  \centering
  \mbox{
    \subfigure[Codif. segmento g]{
      \label{sfg:segmentoG}
      \begin{tabular}[b]{|c|c|c|c|c|}
      \hline
      \multicolumn{4}{|c|}{Input}&\multicolumn{1}{|c|}{Output}\\\hline
      a&b&c&d&segmento g\\\hline
      0&0&0&0&0\\\hline
      0&0&0&1&0\\\hline
      0&0&1&0&1\\\hline
      0&0&1&1&1\\\hline
      0&1&0&0&1\\\hline
      0&1&0&1&1\\\hline
      0&1&1&0&1\\\hline
      0&1&1&1&0\\\hline
      1&0&0&0&1\\\hline
      1&0&0&1&1\\\hline
      1&0&1&0&x\\\hline
      1&0&1&1&x\\\hline
      1&1&0&0&x\\\hline
      1&1&0&1&x\\\hline
      1&1&1&0&x\\\hline
      1&1&1&1&x\\\hline
      \end{tabular}
    }
    \subfigure[Mapa K segmento g: $a+b\bar{c}+\bar{b}c+c\bar{d}$]{
      \label{sfg:KsegmentoG}
      \begin{karnaugh-map}[4][4][1][$cd$][$ab$]
        \manualterms{0,0,1,1,1,1,1,0,1,1,x,x,x,x,x,x}
        \implicantedge{3}{2}{11}{10}
        \implicant{2}{10}
        \implicant{4}{13}
        \implicant{12}{10}
      \end{karnaugh-map}   
    }
  }
  \caption{\label{fg:segmentoG} Codificación de activación del segmento g y mapa K correspondiente.}
\end{figure}

\begin{exercise}
  \label{ex:atm_prod}
  Sean $\mathbf{A}$ y $\mathbf{B}$ álgebras de Boole finitas y $\langle
  a,b\rangle\in A\times B$. Demuestre que son equivalentes las
  siguientes afirmaciones:
  \begin{enumerate}
  \item
    $\langle a,b\rangle\in
    \operatorname{Atm}(\mathbf{A}\times\mathbf{B})$ \label{ex:atm_prod_A}
  \item ($a\in \operatorname{Atm}(\mathbf{A})$ y $b=0$) ó ($a=0$ y
    $b\in \operatorname{Atm}(\mathbf{B})$) \label{ex:atm_prod_B}
  \end{enumerate}
\end{exercise}

\begin{proof}
  Para demostrar que la afirmación \hyperref[ex:atm_prod_A]{\ref*{ex:atm_prod_A})}
  implica a la \hyperref[ex:atm_prod_B]{\ref*{ex:atm_prod_B})},
  demostremos la implicación contrarrecíproca. Con el fin de articular
  el razonamiento abreviemos por:
  \begin{itemize}
  \item $\alpha$ la expresión ``$a\in
    \operatorname{Atm}(\mathbf{A})$''
  \item $\beta$ la expresión ``$b=0$''
  \item $\varphi$ la expresión ``$a=0$'' y
  \item $\psi$ la expresión
    ``$b\in \operatorname{Atm}(\mathbf{B})$''.
  \end{itemize}
  La negación de la afirmación
  \hyperref[ex:atm_prod_B]{\ref*{ex:atm_prod_B})} responde a lo
  siguiente:
  \begin{align*}
    \neg((\alpha\wedge\beta)\vee (\varphi\wedge\psi)&=
        \neg(\alpha\wedge\beta)\wedge\neg (\varphi\wedge\psi)\\
    &=(\neg\alpha\vee\neg\beta)\wedge(\neg\varphi\vee\neg\psi)\\
    &=(\neg\alpha\wedge(\neg\varphi\vee\neg\psi))
    \vee (\neg\beta\wedge(\neg\varphi\vee\neg\psi))\\
    &=(\neg\alpha\wedge\neg\varphi)\vee(\neg\alpha\wedge\neg\psi)
    \vee(\neg\beta\wedge\neg\varphi)\vee(\neg\beta\wedge\neg\psi)
  \end{align*}
Recordamos ahora que para cualquier conjunto de fórmulas
$\Gamma\cup\{\xi,\zeta\}$ se cumple: 
\begin{equation*}
 \operatorname{Con}(\Gamma,\xi\vee\zeta)=
\operatorname{Con}(\Gamma,\xi)\cap\operatorname{Con}(\Gamma,\zeta) 
\end{equation*}
Por lo que, supuesto lo opuesto de la afirmación
\hyperref[ex:atm_prod_B]{\ref*{ex:atm_prod_B})}, bastará con concluir
que
$\langle a,b\rangle\notin
\operatorname{Atm}(\mathbf{A}\times\mathbf{B})$ en cada uno de los
casos que hemos llegado a distinguir. En definitiva, el razonamiento
es por casos según los siguientes (en realidad tres):
\begin{itemize}
\item $a\notin\operatorname{Atm}(\mathbf{A})$ y $a\neq 0$ (por
  $\neg\alpha\wedge\neg\varphi$); si
  $a\notin\operatorname{Atm}(\mathbf{A})$ entonces existe $x\in A$ tal
  que $0<x<a$. Así pues
  $\langle 0,0\rangle<\langle x,b\rangle<\langle a,b\rangle$ y
  así $\langle a,b\rangle$ no es átomo de $\mathbf{A}\times\mathbf{B}$.
\item $b\neq 0$ y $a\neq 0$ (por $\neg\beta\wedge\neg\varphi$); en
  este caso se tiene que
  $\langle 0,0\rangle<\langle a,0\rangle<\langle a,b\rangle$ y
  así $\langle a,b\rangle$ no es átomo de $\mathbf{A}\times\mathbf{B}$.
\item $b\neq 0$ y $b\notin \operatorname{Atm}(\mathbf{B})$
  (por $\neg\beta\wedge\neg\psi$); si
  $b\notin\operatorname{Atm}(\mathbf{B})$ entonces existe $y\in B$ tal
  que $0<y<b$. Así pues
  $\langle 0,0\rangle<\langle a,y\rangle<\langle a,b\rangle$ y
  así $\langle a,b\rangle$ no es átomo de
  $\mathbf{A}\times\mathbf{B}$.
\item $a\notin\operatorname{Atm}(\mathbf{A})$ y
  $b\notin \operatorname{Atm}(\mathbf{B})$ (por
  $\neg\alpha\wedge\neg\psi$); éste es un caso específico de los
  anteriormente tratados, a menos que alguno de los elementos, $a$ ó
  $b$, sea igual a $0$. Si $a=0=b$ concluimos que $\langle a,b\rangle$
  no es un átomo debido a que para serlo es condición necesaria la no
  nulidad. Si $a\neq 0$ pero $b=0$, tomaremos $x\in A$ tal que $0<x<a$
  y entonces
  $\langle 0,0\rangle<\langle x,0\rangle<\langle a,b\rangle$ con lo
  que $\langle a,b\rangle$ no puede ser átomo. Si el caso es $a=0$
  pero $b\neq 0$, un razonamiento análogo nos lleva a que
  $\langle a,b\rangle$ no puede ser átomo.
\end{itemize}
\textbf{Recíprocamente}, supongamos que se cumple lo que afirma
\hyperref[ex:atm_prod_B]{\ref*{ex:atm_prod_B})} y demostremos, como
conclusión, lo que afirma
\hyperref[ex:atm_prod_A]{\ref*{ex:atm_prod_A})}. Supongamos, pues, que
$a\in \operatorname{Atm}(\mathbf{A})$ y que $b=0$. Si
$\langle x,y\rangle\in A\times B$ y
$\langle 0,0\rangle\leq\langle x,y\rangle\leq\langle a,0\rangle$; pero
al ser $a$ átomo se cumple $x=0$ ó $x=a$ por lo que en realidad se
tiene que $\langle 0,0\rangle=\langle x,y\rangle$ o
$\langle x,y\rangle=\langle a,b\rangle$ y de ello que
$\langle a,b\rangle\in
\operatorname{Atm}(\mathbf{A}\times\mathbf{B})$. Si se da que $a=0$ y
$b\in \operatorname{Atm}(\mathbf{B})$, un razonamiento análogo conduce
a que también
$\langle a,b\rangle\in
\operatorname{Atm}(\mathbf{A}\times\mathbf{B})$. Así pues se tiene
demostrada la afirmación
\hyperref[ex:atm_prod_A]{\ref*{ex:atm_prod_A})}.
\end{proof}

\begin{exercise}
 Sea $\mathbf{B}$ un álgebra de Boole
  \textbf{cualquiera}. Demuestre que para todo $a,b,c\in B$ son
  equivalentes las siguientes afirmaciones:
  \begin{enumerate}
  \item $a\oplus b\leq c$
  \item $ac'=bc'$
  \end{enumerate}
\end{exercise}

  \begin{solution}
    Para cualesquiera $a,b,c\in B$,
    \begin{align*}
      ac'(bc')'&=ac'(b'+c'')\\
               &=ac'(b'+c)\\
               &=ac'b'+ac'c\\
               &=ab'c'+a0\\
               &=ab'c'+0\\
               &=ab'c'
    \end{align*}
    Por tanto:
    \begin{equation}
      \label{eq:par18192A}
      ac'(bc')'=ab'c'
    \end{equation}
    y también (como ecuación es idéntica):
    \begin{equation}
      \label{eq:par18192B}
      bc'(ac')'=a'bc'
    \end{equation}
    Supongamos ahora que $a\oplus b\leq c$, o equivaléntemente que
    $ab'+a'b\leq c$. Entonces, por transitividad, $ab'\leq c$ y
    $a'b\leq c$ o lo que es equivalente: $ab'c'=0$ y
    $a'bc'=0$. Entonces, por
    \hyperref[eq:par18192A]{(\ref*{eq:par18192A})}, $ac'(bc')'=0$ y por
    \hyperref[eq:par18192B]{(\ref*{eq:par18192B})}, $bc'(ac')'=0$. De lo
    primero se deduce que $ac'\leq bc'$ y de lo segundo que
    $bc'\leq ac'$. Por la antisimetría de $\leq$, se tiene que
    $ac'=bc'$. \textbf{Recíprocamente}, supongamos que
    $ac'=bc'$. Entonces $ac'(bc')'=0$ y $bc'(ac')'=0$; como antes
    $ab'c'=0=a'bc'$. Deducimos de esto que $ab'\leq c$ y que $a'b\leq
    c$, de donde:
    \begin{equation*}
      a\oplus b = ab'+a'b\leq c
    \end{equation*}
  \end{solution}

  \begin{exercise}
    \label{ex:Zhegalkine}
    Sobre expresiones boolenas $\varphi$ definamos la siguiente
    función $p_{G}$ que propociona porlinomios con coeficientes de
    $\mathbb{Z}_{2}$:
    \begin{equation*}
      \operatorname{p_{G}}(\varphi)=
      \begin{cases}
        0,&\text{ si }\varphi\equiv 0\\
        1,&\text{ si }\varphi\equiv 1\\
        x_{i},&\text{ si }\varphi\equiv x_{i}\\
        \operatorname{p_{G}}(\alpha)+1,
        &\text{ si }\varphi\equiv\alpha'\\
        \operatorname{p_{G}}(\alpha)\operatorname{p_{G}}(\beta)
        +\operatorname{p_{G}}(\alpha)
        +\operatorname{p_{G}}(\beta),&\text{ si }\varphi\equiv\alpha+\beta\\
        \operatorname{p_{G}}(\alpha)\operatorname{p_{G}}(\beta),
        &\text{ si }\varphi\equiv\alpha\cdot\beta
      \end{cases}
    \end{equation*}
    Para cualquier expresión booleana $\varphi$,
    $\operatorname{p_{G}}(\varphi)$ está bien definida (por el
    \textit{principio de lectura única}) y es denominado el
    \href{https://en.wikipedia.org/wiki/Ivan_Ivanovich_Zhegalkin}{\textit{Polinomio
        de Zhegalkine}} de $\varphi$. Encuentre el polinomio de
    Zhegalkin de las siguientes expresiones booleanas:
    \begin{enumerate}
    \item $x\to y$
    \item $x\uparrow y$ (Sheffer stroke con notación, con notación de Bocheński Dxy)
    \item $x\downarrow y$ (Quine dagger or Peirce's arrow, con notación de Bocheński Xxy)
    \item $x\oplus y$
    \item $x\to(y\to z)$
    \item $(x\to y)\to z$
    \end{enumerate}
  \end{exercise}