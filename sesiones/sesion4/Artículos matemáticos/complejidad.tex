%%% Local Variables: 
%%% mode: latex
%%% TeX-master: "bbp_formula"
%%% End:

\section{Complejidad del Método}

Para calcular el enésimo dígito de $\pi$ en base $16$ (ó el $4n$-ésimo
dígito en base2):

\subsection*{Complejidad Temporal}

\begin{itemize}
\item $B_{N}'(a)$; se calcula con complejidad lineal
  $\operatorname{O}(1)$
\item $A_{N}'(a)$; utilizando el método de la exponenciación binaria,
  sus términos se calculan en $\operatorname{O}(\log_{2}(n))$. Así la
  suma de $n$ términos, $A_{N}'(a)$, se calcula en
  $\operatorname{O}(n\log_{2}(n))$.
\end{itemize}
Así $S_{N}'(a)$ se calcula en 
\begin{equation*}
  \operatorname{O}(1)+\operatorname{O}(n\log_{2}(n))=
  \operatorname{O}(n\log_{2}(n))
\end{equation*}
Finalmente, $\pi_{n}$ se calcula en
\begin{equation*}
  4\operatorname{O}(n\log_{2}(n))=\operatorname{O}(n\log_{2}(n))    
\end{equation*}
Así pues, si el tiempo de cálculo es proporcional a $n\log_{2}(n)$,
es casi lineal.

\subsection*{Complejidad Espacial}

La complejidad en el uso de memoria es constante, ya que sólo se
realizan sumas sucesivas de pequeños números (con una precisión de
unos diez decimales es suficiente).