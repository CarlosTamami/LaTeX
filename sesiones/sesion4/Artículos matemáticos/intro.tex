%%% Local Variables: 
%%% mode: latex
%%% TeX-master: "bbp_formula"
%%% End:

\section{Introducción}

La \textit{fórmula de Bailey-Borwein-Plouffe} (o \textit{fórmula BBP})
permite calcular el enésimo dígito de $\pi$ en base $2$ ó $16$ sin
necesidad de hallar los precedentes, de una manera rápida y utilizando
muy poco espacio de memoria en la computadora. \textit{Simon Plouffe}
junto con \textit{David Bailey} y \textit{Peter Borwein} hallaron esta
fórmula el 19 de septiembre de 1995 usando un programa informático
llamado \texttt{PSLQ} que busca relaciones entre números enteros.

La fórmula BBP tiene la siguiente expresión:
\begin{equation*}
  \pi = \sum_{k=0}^{\infty}\frac{1}{16^{k}}
  \left(
    \frac{4}{8k+1}
    -\frac{2}{8k+4}
    -\frac{1}{8k+5}
    -\frac{1}{8k+6}
  \right)
\end{equation*}
La demostración de esta igualdad se encuentra más abajo.