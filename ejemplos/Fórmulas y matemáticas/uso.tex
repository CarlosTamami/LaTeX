%%% Local Variables: 
%%% mode: latex
%%% TeX-master: "bbp_formula"
%%% End:

\section{Uso de la Fórmula para Calcular los Decimales del Número
  $\boldsymbol{\pi}$}


A continuación se muestra el cálculo del enésimo dígito hexadecimal de
$\pi$. Primero se debe observar que el dígito ubicado en la posición
$N+1$ de $\pi$ en base $16$ es el mismo que el primer dígito
hexadecimal de $16^{N}\pi$. En efecto, como en la base $10$,
multiplicar un número en base $16$ por $16$ equivale a desplazar la
coma decimal un lugar hacia la derecha. De esta manera, multiplicando
por $16^{N}$ la coma se desplaza $N$ lugares hacia la derecha. El
problema original se reduce al cálculo del primer dígito de
$16^{N}\pi$. Usando la \hyperref[eq:bbp]{fórmula BBP }:
\begin{equation*}
  16^{N}\pi =\sum_{k=0}^{\infty}16^{N-k}
  \left(
    \frac{4}{8k+1}
    -\frac{2}{8k+4}
    -\frac{1}{8k+5}
    -\frac{1}{8k+6}
  \right) \nonumber
\end{equation*}
    
El cálculo de los primeros dígitos hexadecimales a la derecha de la
coma de este número no es sencillo por dos razones: el número es muy
grande y la suma es infinita.
  
\begin{definition}
  Sea $N$ un número natural no nulo y $a$ un número natural. Definimos
  $S_{N}(a)$, $A_{N}(a)$ y $B_{N}(a)$ por las siguientes igualdades:
  \begin{eqnarray*}
    S_{N}(a)&=&\sum_{k=0}^{\infty}\frac{16^{N-k}}{8k+a}\\
    A_{N}(a)&=&\sum_{k=0}^{N-1}\frac{16^{N-k}}{8k+a}\\
    B_{N}(a)&=&\sum_{k=N}^{\infty}\frac{16^{N-k}}{8k+a}\\
  \end{eqnarray*}
\end{definition}

\begin{remark}
  Para tódo número natural $N$ no nulo y todo número natural:
  \begin{equation*}
    S_{N}(a) = A_{N}(a)+B_{N}(a)
  \end{equation*}
\end{remark}


El cálculo de los primeros dígitos hexadecimales de $S_{N}(a)$
permitirá obtener los de $16^{N}\pi$, a través de la relación:
\begin{equation*}
  16^{N}\pi=4S_{N}(1)-2S_{N}(4)-S_{N}(5)-S_{N}(6)
\end{equation*}

\subsection{Cálculo de $\boldsymbol{B_{N}(a)}$}

Aunque $B_{N}(a)$ es una suma infinita, es muy fácil de calcular
porque sus términos son pequeños y decrecen rápidamente. En efecto:
\begin{itemize}
\item el prmer término de la suma es $b_{N}=1/(8N+a)$. Si se busca
  el enésimo dígito hexadecimal de $\pi$ (digamos, por ejemplo,
  $N=1000000000$), el primer término es mucho menor que $1$.
\item Además, cada término tiene un cero más a la derecha de la coma
  que el precedente, porque para $k\geq N$, $b_{k}>16b_{k+1}$:
  \begin{align*}
    \frac{b_{k}}{b_{k+1}}&=\frac{16^{N-k}}{16^{N-(k+1)}}\frac{8(k+1)+a}{8k+a}\\
                         &=16\left(1+\frac{8}{8k+a}\right) \longrightarrow 16^{+}
  \end{align*}
\end{itemize}

Finalmente, la suma $B_{N}(a)$ es de la forma (en el peor caso):
\begin{align*}
  B_{N}&=0,**********\ldots\\
       &+0,0*********\ldots\\
       &+0,00********\ldots\\
       &+0,000*******\ldots\\ 
\end{align*}

Por lo tanto, para obtener $B_{N}(a)$ con una precisión de $P$ cifras
detrás de la coma, es suficiente calcular los $P$ primeros términos de
la suma, agregándose algunos términos más para eviatar errores que
aparecen al realizar cálculos con valores aproximados. Así, se
calcula:
\begin{equation*}
  B_{N}'=\sum_{k=N}^{N+P+10}\frac{16^{N-k}}{8k+a} 
\end{equation*}
Como esta suma posee una pequeña cantidad de términos, el tiempo que
insume esta operación es insignificante para la computadora.

\subsection{Cálculo de $\boldsymbol{A_{N}(a)}$}

El problema para el cálculo de $A_{N}(a)$ es que los primeros términos
son muy grandes ($N$ cifras de base $16$ antes de la coma). Sin
embargo, al igual que las primeras cifras detrás de la coma, no
importa la parte entera, que también es grande. Por lo tanto, puede
``eliminarse'' usando aritmética modular.

Toda la dificultad se reduce a hallar la parte fraccionaria de
$16^{N-k}/(8k+a)$. Para ello realizamos la división entera de
$16^{N-k}$ entre $(8k+a)$. Sabemos que existen números enteros $q$
y $r$ tales que:
\begin{itemize}
\item $16^{N-k}=q(8k+a)+r$
\item $0\leq r<8k+a$
\end{itemize}
Por tanto:
\begin{equation*}
  \frac{16^{N-k}}{8k+a}=q+\frac{r}{8k+a}
\end{equation*}
siendo $r/(8k+a)<1$ y por tanto es la parte fraccionaria de
$16^{N-k}/(8k+a)$ y
\begin{equation*}
  \frac{r}{8k+a}=\frac{16^{N-k}\mod (8k+a)}{8k+a}
\end{equation*}

Utilizando el método de la exponenciación binaria,
$16^{N-k}\mod (8k+a)$ se calcula rápidamente (con un tiempo de
ejecución de $\operatorname{O}(\log_{2}(N-k))$.

\subsection{Conclusión}

Al fin y al cabo, para obtener los primeros dígitos de $\pi$ en base
$16$ (ó $2$) se deben calcular los primeros dígitos de:
\begin{equation*}
  \pi_{N}=4S_{N}'(1)-2S_{N}'(4)-S_{N}'(5)-S_{N}'(6)
\end{equation*}
donde
\begin{equation*}
  S_{N}'(a) = \sum_{k=0}^{N-1}\frac{16^{N-k}\mod (8k+a)}{8k+a}
  +\sum_{k=N}^{N+P+10}\frac{16^{N-k}}{8k+a}
\end{equation*}