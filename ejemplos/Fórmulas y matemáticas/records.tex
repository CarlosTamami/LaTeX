%%% Local Variables: 
%%% mode: latex
%%% TeX-master: "bbp_formula"
%%% End:

\section{Los Récords}

\begin{flushleft}
  Para poder comparar, hasta el año 2008 se habían obtenido los
  primeros $1,241$ billones de decimales de $\pi$ (aproximadamente
  $4,123$ billones de bits):
  \begin{itemize}
  \item 7 de octubre de 1996
    (\href{https://es.wikipedia.org/wiki/Fabrice_Bellard}{Fabrice
      Bellard}): dígito número $400$ mil millones en base $2$.
  \item septiembre de 1997
    (\href{https://es.wikipedia.org/wiki/Fabrice_Bellard}{Fabrice
      Bellard}): billonésimo dígito en base $2$.
  \item febrero de 1999 (Colin Percival): dígito número $40$ billones en
    base $2$
  \item 2001: dígito número $4000$ billones en base $2$
  \item 2011: dígito número $10$ billones en base $2$
  \end{itemize}
\end{flushleft}

\section{Cálculo del enésimo decimal}

\begin{flushleft}
  Actualmente, no existe ninguna fórmula eficaz para hallar el enésimo
  decimal de $\pi$ en base $10$. \textit{Simon Plouffe} ha
  desarrollado en diciembre de 1996, a partir de una serie muy antigua
  que calcula $\pi$ basado en los coeficientes de
  \href{https://es.wikipedia.org/wiki/Binomio_de_Newton}{binomio de
    Newton}, un método para calcular cifras aisladas base $10$, pero
  debido a su complejidad $\operatorname{O}(n^{3}\log_{2}(n))$ pierde
  su utilidad en la
  práctica. \href{https://es.wikipedia.org/wiki/Fabrice_Bellard}{Fabrice
    Bellard} ha mejorado el algoritmo para alcanzar un nivel de
  complejidad en $\operatorname{O}(n^{2})$, pero no es suficiente para
  competir con los métodos convencionales que calculan todos los
  decimales.
\end{flushleft}