\section{Álgebra de Boole}

% https://www.gatevidyalay.com/neutral-functions/
% https://www.gatevidyalay.com/self-dual-functions-dual-of-a-boolean-expression/

\begin{definition}
  Sea $\varphi$ una expresión booleana. $\varphi$ es autodual sii, por
  definición, $\varphi^{d}=\varphi$.
\end{definition}

\begin{exercise}
  Compruebe si la expresión booleana:
  \begin{equation*}
    \varphi=xy+yz+zx
  \end{equation*}
  es autodual.
  %https://math.stackexchange.com/questions/2493779/self-dual-functions
\end{exercise}

\begin{solution}
  Lo es. En efecto:
  \begin{align*}
    \varphi^{d}&=(x+y)(y+z)(z+x)\\
               &=x(y+z)(z+x)+y(y+z)(z+x)\\
               &=(xy+xz)(z+x)+(yy+yz)(z+x)\\
               &=(xy+xz)z+(xy+xz)x+(y+yz)z+(y+yz)x\\
               &=xyz+xzz+xyx+xzx+yz+yzz+yx+yzx\\
               &=xyz+xz+xy+xz+yz+yz+yx+yzx\\
               &=xz+yz+yx+yzx\\
               &=(xy+xyz)+yz+zx\\
               &=xy+yz+zx\\
  \end{align*}
  De donde, $\varphi^{d}=\varphi$ y esto revela la autodualidad.
\end{solution}

\begin{theorem}
  Sea $\varphi$ una expresión booleana. Son equivalentes las
  siguientes afirmaciones:
  \begin{enumerate}
  \item $\varphi$ es autodual.
  \item $f_{\varphi}(x_{1},\ldots,x_{n})=f(x_{1}',\ldots,x_{n}')'$
  \end{enumerate}
\end{theorem}

\begin{exercise}
  Demuestre que si $\varphi$ y $\psi$ son expresiones booleanas tales
  que $\varphi=\psi$, entonces son equivalentes:
  \begin{enumerate}
  \item $\varphi$ es autodual.
  \item $\psi$ es autodual.
  \end{enumerate}
\end{exercise}

\begin{solution}
  Supongamos que $\varphi=\psi$ y que $\varphi$ es autodual. Entonces:
  \begin{align*}
    \psi^{d}&=\varphi^{d}\\
            &=\varphi\\
            &=\psi
  \end{align*}
  Esto es suficiente.
\end{solution}

\begin{definition}
  Una función booleana es autodual sii, por def., al menos una de
  las expresiones booleanas que la engendran (y por tanto todas) es
  autodual.
\end{definition}

\begin{definition}
  \index{función! booleana neutra} Una función booleana es
  \textit{neutra} sii, por definición, viene descrita por el mismo
  número de minterm que de maxterm.
\end{definition}

\begin{exercise}
  ¿Cuántas funciones booleanas neutras de $n$ símbolos de variable hay?
\end{exercise}

\begin{solution}
  Si ha de tener $n$ símbolos de variable, hay $2^{n}$ minterm. Si ha
  de ser neutra, vendrá descrita por $2^{n-1}$ minterm. La pregunta es
  cuántas posibles elecciones de estos $2^{n-1}$ minterm pueden
  hacerse entre los $2^{n}$ minterm. Dada la conmutatividad de la suma
  en las expresiones, el número buscado es $C(2^{n},2^{n-1})$, es decir:
  \begin{equation*}
    \binom{2^{n}}{2^{n-1}}
  \end{equation*}
\end{solution}

\begin{exercise}
  \label{exer:neutras2}
  Diga el número de funciones booleanas neutras con dos símbolos de
  variable y encuentre todas ellas:
\end{exercise}

\begin{solution}
  El número de tales funciones booleanas es:
  \begin{align*}
    C(2^{n},2^{n-1})&=C(4,2)\\
                    &=\binom{4}{2}\\
                    &=\frac{4!}{2!2!}\\
                    &=\frac{24}{4}\\
                    &=6
  \end{align*}
  Las funciones son:
  \begin{align*}
    h_{0}(x,y)&=\sum m(0,1)=x'y'+x'y=x'(y'+y)=x'\\
    h_{1}(x,y)&=\sum m(0,2)=x'y'+xy'=y'\\
    h_{2}(x,y)&=\sum m(0,3)=x'y'+xy=x\equiv y\\
    h_{3}(x,y)&=\sum m(1,2)=x'y+xy'=x\oplus y\\
    h_{4}(x,y)&=\sum m(1,3)=x'y+xy=y\\
    h_{5}(x,y)&=\sum m(2,3)=xy'+xy=x
  \end{align*}
\end{solution}

\begin{definition}
  Dado un minterm, su minterm mutuamente exclusivo es el obtenido
  del primero cambiando cada una de los símbolos de variable que lo
  forman por el complemento del mismo.
\end{definition}

\begin{lemma}
  Sean $m_{1}$ y $m_{2}$ minterm de $n$ símbolos de variable
  ($1\leq n$), con respresentaciones decimales respectivas:
  $k_{m_{1}}$ y $k_{m_{2}}$. Son equivalentes las siguientes
  afirmaciones:
  \begin{enumerate}
  \item $\langle m_{1},m_{2}\rangle$ es una pareja de minterm
    mutuamente exclusivos.
  \item $k_{m_{1}}+k_{m_{2}}=2^{n}-1$.
  \end{enumerate}
\end{lemma}

\begin{example}
  Las siguientes parejas son de minterm mutuamente exclusivos:
  \begin{itemize}
  \item $\langle xyz,x'y'z'\rangle$; $k_{xyz}=7$, $k_{x'y'z'}=0$,
    $7+0=7=8-1=2^{3}-1$. 
  \item $\langle xy'z,x'yz'\rangle$; $k_{xyz}=5$, $k_{x'y'z'}=2$,
    $5+2=7=8-1=2^{3}-1$.
  \end{itemize}
\end{example}

\begin{theorem}
  Sea $\varphi$ una expresión booleana en la que ocurren $n$ símbolos
  de variable, siendo $n$ cualquier número natural no nulo. Son
  equivalentes las siguientes afirmaciones:
  \begin{enumerate}
  \item $\varphi$ es autodual.
  \item $f_{\varphi}$ es neutra y ninguna pareja de los minterm que la
    describen es una pareja de minterm mutuamente exclusivos.
  \end{enumerate}
\end{theorem}

\begin{exercise}
  ¿Cuántas expresiones autoduales con $n$ símbolos de variable hay?
  % https://www.gatevidyalay.com/self-dual-functions-dual-of-a-boolean-expression/
\end{exercise}

\begin{solution}
  Una expresión autodual debe ser neutra y para ello debe ocurrir que
  el número de sus minterm sea igual al de sus maxterm. Así pues,
  estará representada por la mitad de los minterm para ese número de
  símbolos de varialbe, i.e.  $2^{n-1}=2^{n}/2$ términos. Ahora bien,
  para cada uno de esos minterm tenemos $2$ opciones: incluirlo (y
  entonces no estará incluido su minterm mutuamente exclusivo) o no (y
  entonces estará incluido su minterm mutuamente exclusivo); así pues,
  el número de expresiones autoduales es $2^{2^{n-1}}$.
\end{solution}

\begin{exercise}
  Diga el número de expresiones booleanas autoduales con dos símbolos de
  variable y encuentre todas ellas:
\end{exercise}

\begin{solution}
  El número de tales funciones booleanas es
  $2^{2^{2-1}}=2^{2}=4$. Dichas expresiones estarán entre las
  expresiones con dos símbolos de variable que engendran funciones neutras
  con dos símbolos de variable. Estas fueron determinadas en el
  \hyperref[exer:neutras2]{Ejercicio \ref*{exer:neutras2}} y eran: 
  \begin{align*}
    h_{0}(x,y)&=\sum m(0,1)=x'y'+x'y=x'(y'+y)=x'\\
    h_{1}(x,y)&=\sum m(0,2)=x'y'+xy'=y'\\
    h_{2}(x,y)&=\sum m(0,3)=x'y'+xy=x\equiv y\\
    h_{3}(x,y)&=\sum m(1,2)=x'y+xy'=x\oplus y\\
    h_{4}(x,y)&=\sum m(1,3)=x'y+xy=y\\
    h_{5}(x,y)&=\sum m(2,3)=xy'+xy=x
  \end{align*}
  pero de ellas hemos de excluir $h_{2}(x,y)=\sum m(0,3)$, pues
  $0+3=3=2^{2}-1$ y $h_{3}(x,y)=\sum m(1,2)$ pues $1+2=3=2^{2}-1$. Así
  pues, las expresiones autoduales sobre dos símbolos de variable son:
  \begin{itemize}
  \item $x'$
  \item $y'$
  \item $x$
  \item $y$
  \end{itemize}
\end{solution}

\begin{exercise}
  Componga un programa en \texttt{Python} o \texttt{Haskell} que dado
  un número natural no nulo $n$, encuentre (por al menos una expresión
  que las engendre) todas y cada una de las funciones booleanas de $n$
  variables que sean autoduales.
\end{exercise}

\begin{exercise}
  De las siguientes funciones:
  \begin{enumerate}
  \item $f(x,y,z)=\sum m(0,2,3)$\\
  \item $g(x,y,z)=\sum m(0,1,6,7)$\\
  \item $h(x,y,z)=\sum m(0,1,2,4)$\\
  \item $k(x,y,z)=\sum m(3,5,6,7)$\\
  \end{enumerate}
  diga cuales están realizadas por expresiones autoduales.
\end{exercise}

\begin{solution}
  El número de minterm posibles para expresar funciones de $3$
  símbolos de variable son $8$. Esto excluye de las consideraciones a
  $f(x,y,z)$, que no puede provenir de una expresión autodual al ser
  suma de $3$ minterm y no de $4$. Para el resto, tengamos en cuenta
  que $2^{3}-1=7$; así pues:
  \begin{itemize}
  \item $g(x,y,z)$ no puede ser autodual, pues por ejemplo $6+1=7$
    aunque también $0+7=7$.
  \item $h(x,y,z)$ es autodual, pues: viene descrita por $4$ minterm y
    $0+1\neq 7$, $0+2\neq 7$, $0+4\neq 7$, $1+2\neq 7$, $1+4\neq 7$ y
    $2+4\neq 7$.
  \item $k(x,y,z)$ es autodual, pues: viene descrita por $4$ minterm y
    $3+5\neq 7$, $3+6\neq 7$, $3+7\neq 7$, $5+6\neq 7$, $5+7\neq 7$ y
    $6+7\neq 7$.
  \end{itemize}
\end{solution}

\begin{exercise}
  Compruebe si la expresión booleana:
  \begin{equation*}
    \varphi=xy+yz+zx
  \end{equation*}
  es autodual.
\end{exercise}

\begin{solution}
  El número de símbolos de variable utilizado para describir $\varphi$
  es $3$. Tenemos que $2^{3-1}=4$ y que $2^{3}-1=7$. Por otra parte,
  \begin{equation*}
    f_{\varphi}(x,y,z)=\sum m(3,5,6,7)
  \end{equation*}
  Dado que $f_{\varphi}$ viene expresada por $4$ minterm y que ninguna
  pareja de ellos es de miterm mutuamente exclusivo (basta observar
  que $3+5>7$), concluimos que $f_{\varphi}$ es autodual y que también
  lo es $\varphi$.
\end{solution}

\begin{exercise}
  Componga un programa en \texttt{Python} o \texttt{Haskell} que dada
  una expresión booleana, decida si es o no autodual.
\end{exercise}

\begin{exercise}
  En el álgebra de Boole $\mathbf{D(67830)}$ calcule:
  \begin{enumerate}
  \item sus átomos
  \item $1615\cdot 2261$
  \item $1615+2261$
  \item $399'$
  \end{enumerate}
\end{exercise}

\begin{solution}~
  \begin{enumerate}
  \item Como $67830=2\cdot 3\cdot 5\cdot 7\cdot 17\cdot 19$,
    constatamos que efectivamente $\mathbf{D(67830)}$ es un álgebra de
    Boole y que
    \begin{equation*}
      \operatorname{Atm}(\mathbf{D(67830)})=
      \{2,3,5,7,17,19\}
    \end{equation*}
  \item $1615\cdot 2261=(1615,2261)=323$
  \item $1615+2261=[1615,2261]=\frac{1615\cdot 2261}{323}=11305$
  \item $399'=\frac{67830}{399}=170$
  \end{enumerate}
\end{solution}

\begin{exercise}
  Sean $m$ y $n$ números naturales no nulos tales que $\mathbf{D(m)}$
  y $\mathbf{D(n)}$ son álgebras de Boole no triviales. Demuestre que
  son equivalentes las siguientes afirmaciones:
  \begin{enumerate}
  \item $\mathbf{D(mn)}$ es un álgebra de Boole.
  \item $(m,n)=1$.
  \end{enumerate}
  Demuestre además que si $m$ y $n$ son números naturales no nulos
  tales que $\mathbf{D(m)}$, $\mathbf{D(n)}$ y $\mathbf{D(mn)}$ son
  álgebras de Boole, entonces $\mathbf{D(m)}\times\mathbf{D(n)}$ y
  $\mathbf{D(mn)}$ son álgebras de Boole isomorfas.
\end{exercise}
  
  \begin{solution}
    Sean $m$ y $n$ números naturales no nulos tales que
    $\mathbf{D(m)}$ y $\mathbf{D(n)}$ son álgebras de Boole no
    triviales. Supongamos $\mathbf{D(mn)}$ es un álgebra de Boole y
    que es isomorfa a $\mathbf{D(m)}\times\mathbf{D(n)}$; en tal caso
    $\mathbf{D(mn)}$ es un álgebra de Boole. Por tanto, en la
    factorización de $mn$ como producto de números primos no figura
    ningún primo elevado a una potencia superior a $1$; de donde $m$ y
    $n$ no tienen ningún factor primo en común, o sea,
    $(m,n)=1$. Recíprocamente, supongamos que $(m,n)=1$. Por la
    hipótesis general $P(m)$ y $P(n)$ valen; donde $P(r)$ es la
    afirmación ``existe un número natural $k_{r}$ tal que:
    $r=\prod_{i=0}^{k_{r}}p_{i}$; para todo $0\leq i\leq k_{r}$,
    $p_{i}$ es primo y $p_{i}\neq p_{j}$, siempre que
    $0\leq i<j\leq k_{r}$''. Si $(m,n)=1$, $P(mn)$ vale; por lo que
    $\mathbf{D(mn)}$ es un álgebra de Boole. Por otra parte,
    supongamos que $\mathbf{D(m)}$ y $\mathbf{D(n)}$ son álgebras de
    Boole no triviales. Entonces:
    \begin{itemize}
    \item Los átomos del álgebra de Boole
      $\mathbf{D(m)}\times\mathbf{D(n)}$ son los elementos de
      \begin{equation*}
        \{\langle p,1\rangle\colon p\text{ es primo y }p\mid m\}
        \cup
        \{\langle 1,q\rangle\colon p\text{ es primo y }q\mid n\}
      \end{equation*}
      por lo que posee $k_{m}+k_{n}+2$ átomos.
    \item En las hipótesis, $\mathbf{D(mn)}$ es un álgebra de Boole no
      trivial y su número de átomos es también $k_{m}+k_{n}+2$.
    \end{itemize}
    Al ser $\mathbf{D(m)}\times\mathbf{D(n)}$ y $\mathbf{D(mn)}$
    álgebras de Boole no triviales y tener el mismo número de átomos,
    deben ser isomorfas. Pero si alguna de ellas, o las dos, son
    triviales es resultado es a su vez evidente.
  \end{solution}

\begin{exercise}
  \label{ex:leqEquivalent}
  Sea $\mathbf{B}=\langle B,+,\cdot,\bar{\ },0,1\rangle$ un álgebra de
  Boole. Demuestre que para todo $x,y\in B$ son equivalentes las
  siguientes afirmaciones:
  \begin{enumerate}
  \item $x\leq y$
  \item $x+y=y$
  \item $x\supset y=1$
  \end{enumerate}
\end{exercise}

\begin{solution}
  Sean $x,y\in B$ cualesquiera y supongamos que $x\leq y$, es decir,
  que $xy=x$. Entonces:
  \begin{align*}
    x+y&=xy+y\\
       &=y+yx\\
       &=y
  \end{align*}
Supongamos ahora que $x+y=y$; entonces:
\begin{align*}
  x\supset y&=x\supset(x+y)\\
        &=x\supset(x'\supset y)\\
        &=x'+x+y\\
        &=1+y\\
        &=1
\end{align*}
Finalmente, supongamos que $x\supset y=1$; entonces:
\begin{align*}
  x&=x1\\
   &=x(x\supset y)\\
   &=x(x'+y)\\
   &=xx'+xy\\
   &=0+xy\\
   &=xy
\end{align*}
\end{solution}

\begin{remark}
  En el ambiente del \hyperref[ex:leqEquivalent]{Ejercicio
    \ref*{ex:leqEquivalent}}, supongamos que $xy=x$; entonces:
  \begin{align*}
    x\supset y&=xy\supset y\\
          &=x\supset(y\supset y)\\
          &=x\supset 1\\
          &=1
  \end{align*}
\end{remark}

\begin{exercise}
  \label{ex:nandConsequence}
  Sea $\mathbf{B}=\langle B,+,\cdot,\bar{\ },0,1\rangle$ un álgebra de
  Boole. Demuestre que para todo $x,y\in B$, si $x\uparrow y=0$
  entonces $x=1=y$ 
\end{exercise}

\begin{solution}
  Sean $x,y\in B$ tales que $x\uparrow y=0$, o sea, $0=(xy)'$. Esto es
  equivalente a que $xy=1$, es decir, $x=1=y$.
\end{solution}

\begin{exercise}
  Sea $n$ un número natural y sean
  $f,g,k,h\colon B_{2}^{n}\longrightarrow B_{2}$. Considere la
  función:
  \begin{equation*}
    h(x_{0},\ldots,x_{n-1})=
    \begin{cases}
      f(x_{0},\ldots,x_{n-1}),&\text{ si } k(x_{0},\ldots,x_{n-1})=1\\
      g(x_{0},\ldots,x_{n-1}),&\text{ si } k(x_{0},\ldots,x_{n-1})=0
    \end{cases}
  \end{equation*}
  y demuestre que 
  \begin{equation}
    \label{eq:2}
    h=kf+\bar{k}g
  \end{equation}
  Seguidamente considere el caso
  particular de $n=4$, $f,g,h\colon B_{2}^{4}\longrightarrow B_{2}$
  definidas por: $f(x,y,z)=x\uparrow z$, $g(x,y,z)=\bar{y}$ y $h$
  según lo siguiente
  \begin{equation*}
    h(x,y,z,t)=
    \begin{cases}
      f(x,y,z),& \text{ si }\bar{x}\leq\bar{t}\\
      g(y,z,t),& \text{ si }x<t 
    \end{cases}
  \end{equation*}
  Aplique \hyperref[eq:2]{(\ref*{eq:2})} a este caso para obtener:
  \begin{equation}
    \label{eq:3}
    h(x,y,z,t)=(t\supset x)(x\uparrow z)+\bar{z}
  \end{equation}
  \end{exercise}

  \begin{solution}
    Supongamos que para
    $\langle x_{0},\ldots,x_{n-1}\rangle\in B_{2}^{n}$ se tiene
    $k(x_{0},\ldots,x_{n-1})=1$; entonces
    \begin{align*}
      h(x_{0},\ldots,x_{n-1})&=k(x_{0},\ldots,x_{n-1})f(x_{0},\ldots,x_{n-1})
                            +\bar{k}(x_{0},\ldots,x_{n-1})g(x_{0},\ldots,x_{n-1})\\
                           &=1f(x_{0},\ldots,x_{n-1})
                            +0g(x_{0},\ldots,x_{n-1})\\
                           &=f(x_{0},\ldots,x_{n-1})
    \end{align*}
    y si $k(x_{0},\ldots,x_{n-1})=0$ se tiene análoga e evidentemente
    que $h(x_{0},\ldots,x_{n-1})=g(x_{0},\ldots,x_{n-1})$. Por otra
    parte, $\bar{x}\leq\bar{t}$ es equivalente a $t\leq x$ y, según el
    \hyperref[ex:leqEquivalent]{Ejercicio \ref*{ex:leqEquivalent}},
    esto equivalente a $t\supset x=1$; por tanto, aquí
    $k(x,y,z,t)=t\supset x$. La particularización de
    \hyperref[eq:2]{(\ref*{eq:2})} es:
    \begin{equation*}
      (t\supset x)(x\uparrow z)+\bar{z}
    \end{equation*}
    pues: si $t\supset x=0$ entonces
    $(t\supset x)(x\uparrow z)+\bar{z}=\bar{z}$ y si $t\supset x=1$, caben dos
    casos:
    \begin{itemize}
    \item $x\uparrow z=1$; entonces 
      \begin{align*}
        (t\supset x)(x\uparrow z)+\bar{z}&=1+\bar{z}\\
                       &=1\\
                       &=x\uparrow z
      \end{align*}
    \item $x\uparrow z=0$; entonces
      (cfr. \hyperref[ex:nandConsequence]{Ejercicio
        \ref*{ex:nandConsequence}}) $\bar{z}=0$, luego:
      \begin{align*}
        (t\supset x)(x\uparrow z)+\bar{z}&=0+\bar{z}\\
                       &=0+0\\
                       &=0\\
                       &=x\uparrow z
      \end{align*}
    \end{itemize}
    de lo que deducimos que si $t\supset x=1$ entonces
    $(t\supset x)(x\uparrow z)+\bar{z}=x\uparrow z$.
  \end{solution}

\begin{exercise}
  Sea $\mathbf{B}$ un álgebra de Boole
  \textbf{cualquiera}. Demuestre que para todo $a,b,c\in B$ son
  equivalentes las siguientes afirmaciones:
  \begin{enumerate}
  \item $a\oplus b\leq c$
  \item $ac'=bc'$
  \end{enumerate}
\end{exercise}

\begin{solution}
  Para cualesquiera $a,b,c\in B$,
  \begin{align*}
    ac'(bc')'&=ac'(b'+c'')\\
             &=ac'(b'+c)\\
             &=ac'b'+ac'c\\
             &=ab'c'+a0\\
             &=ab'c'+0\\
             &=ab'c'
  \end{align*}
  Por tanto:
  \begin{equation}
    \label{eq:par18192A}
    ac'(bc')'=ab'c'
  \end{equation}
  y también (como ecuación es idéntica):
  \begin{equation}
    \label{eq:par18192B}
    bc'(ac')'=a'bc'
  \end{equation}
  Supongamos ahora que $a\oplus b\leq c$, o equivaléntemente que
  $ab'+a'b\leq c$. Entonces, por transitividad, $ab'\leq c$ y
  $a'b\leq c$ o lo que es equivalente: $ab'c'=0$ y
  $a'bc'=0$. Entonces, por
  \hyperref[eq:par18192A]{(\ref*{eq:par18192A})}, $ac'(bc')'=0$ y por
  \hyperref[eq:par18192B]{(\ref*{eq:par18192B})}, $bc'(ac')'=0$. De lo
  primero se deduce que $ac'\leq bc'$ y de lo segundo que
  $bc'\leq ac'$. Por la antisimetría de $\leq$, se tiene que
  $ac'=bc'$. \textbf{Recíprocamente}, supongamos que
  $ac'=bc'$. Entonces $ac'(bc')'=0$ y $bc'(ac')'=0$; como antes
  $ab'c'=0=a'bc'$. Deducimos de esto que $ab'\leq c$ y que $a'b\leq
  c$, de donde:
  \begin{equation*}
    a\oplus b = ab'+a'b\leq c
  \end{equation*}
\end{solution}

\begin{exercise}
  \label{it:atomosProd}
  Sean $\mathbf{A}$ y $\mathbf{B}$ álgebras
  de Boole finitas y $\langle a,b\rangle\in A\times B$. Demuestre que
  son equivalentes las siguientes afirmaciones:
  \begin{enumerate}
  \item
    $\langle a,b\rangle\in
    \operatorname{Atm}(\mathbf{A}\times\mathbf{B})$ \label{ex:atm_prod_A}
  \item ($a\in \operatorname{Atm}(\mathbf{A})$ y $b=0$) ó ($a=0$ y
    $b\in \operatorname{Atm}(\mathbf{B})$) \label{ex:atm_prod_B}
  \end{enumerate}
  \end{exercise}

\begin{solution}
  Para demostrar que la afirmación \hyperref[ex:atm_prod_A]{\ref*{ex:atm_prod_A})}
  implica a la \hyperref[ex:atm_prod_B]{\ref*{ex:atm_prod_B})},
  demostremos la implicación contrarrecíproca. Con el fin de articular
  el razonamiento abreviemos por:
  \begin{itemize}
  \item $\alpha$ la expresión ``$a\in
    \operatorname{Atm}(\mathbf{A})$''
  \item $\beta$ la expresión ``$b=0$''
  \item $\varphi$ la expresión ``$a=0$'' y
  \item $\psi$ la expresión
    ``$b\in \operatorname{Atm}(\mathbf{B})$''.
  \end{itemize}
  La negación de la afirmación
  \hyperref[ex:atm_prod_B]{\ref*{ex:atm_prod_B})} responde a lo
  siguiente:
  \begin{align*}
    \neg((\alpha\wedge\beta)\vee (\varphi\wedge\psi))&=
        \neg(\alpha\wedge\beta)\wedge\neg (\varphi\wedge\psi)\\
    &=(\neg\alpha\vee\neg\beta)\wedge(\neg\varphi\vee\neg\psi)\\
    &=(\neg\alpha\wedge(\neg\varphi\vee\neg\psi))
    \vee (\neg\beta\wedge(\neg\varphi\vee\neg\psi))\\
    &=(\neg\alpha\wedge\neg\varphi)\vee(\neg\alpha\wedge\neg\psi)
    \vee(\neg\beta\wedge\neg\varphi)\vee(\neg\beta\wedge\neg\psi)
  \end{align*}
Recordamos ahora que para cualquier conjunto de fórmulas
$\Gamma\cup\{\xi,\zeta\}$ se cumple: 
\begin{equation*}
 \operatorname{Con}(\Gamma,\xi\vee\zeta)=
\operatorname{Con}(\Gamma,\xi)\cap\operatorname{Con}(\Gamma,\zeta) 
\end{equation*}
Por lo que, supuesto lo opuesto de la afirmación
\hyperref[ex:atm_prod_B]{\ref*{ex:atm_prod_B})}, bastará con concluir
que
$\langle a,b\rangle\notin
\operatorname{Atm}(\mathbf{A}\times\mathbf{B})$ en cada uno de los
casos que hemos llegado a distinguir. En definitiva, el razonamiento
es por casos según los siguientes (en realidad tres):
\begin{itemize}
\item $a\notin\operatorname{Atm}(\mathbf{A})$ y $a\neq 0$ (por
  $\neg\alpha\wedge\neg\varphi$); si
  $a\notin\operatorname{Atm}(\mathbf{A})$ entonces existe $x\in A$ tal
  que $0<x<a$. Así pues
  $\langle 0,0\rangle<\langle x,b\rangle<\langle a,b\rangle$ y
  así $\langle a,b\rangle$ no es átomo de $\mathbf{A}\times\mathbf{B}$.
\item $b\neq 0$ y $a\neq 0$ (por $\neg\beta\wedge\neg\varphi$); en
  este caso se tiene que
  $\langle 0,0\rangle<\langle a,0\rangle<\langle a,b\rangle$ y
  así $\langle a,b\rangle$ no es átomo de $\mathbf{A}\times\mathbf{B}$.
\item $b\neq 0$ y $b\notin \operatorname{Atm}(\mathbf{B})$
  (por $\neg\beta\wedge\neg\psi$); si
  $b\notin\operatorname{Atm}(\mathbf{B})$ entonces existe $y\in B$ tal
  que $0<y<b$. Así pues
  $\langle 0,0\rangle<\langle a,y\rangle<\langle a,b\rangle$ y
  así $\langle a,b\rangle$ no es átomo de
  $\mathbf{A}\times\mathbf{B}$.
\item $a\notin\operatorname{Atm}(\mathbf{A})$ y
  $b\notin \operatorname{Atm}(\mathbf{B})$ (por
  $\neg\alpha\wedge\neg\psi$); éste es un caso específico de los
  anteriormente tratados, a menos que alguno de los elementos, $a$ ó
  $b$, sea igual a $0$. Si $a=0=b$ concluimos que $\langle a,b\rangle$
  no es un átomo debido a que para serlo es condición necesaria la no
  nulidad. Si $a\neq 0$ pero $b=0$, tomaremos $x\in A$ tal que $0<x<a$
  y entonces
  $\langle 0,0\rangle<\langle x,0\rangle<\langle a,b\rangle$ con lo
  que $\langle a,b\rangle$ no puede ser átomo. Si el caso es $a=0$
  pero $b\neq 0$, un razonamiento análogo nos lleva a que
  $\langle a,b\rangle$ no puede ser átomo.
\end{itemize}
\textbf{Recíprocamente}, supongamos que se cumple lo que afirma
\hyperref[ex:atm_prod_B]{\ref*{ex:atm_prod_B})} y demostremos, como
conclusión, lo que afirma
\hyperref[ex:atm_prod_A]{\ref*{ex:atm_prod_A})}. Supongamos, pues, que
$a\in \operatorname{Atm}(\mathbf{A})$ y que $b=0$. Si
$\langle x,y\rangle\in A\times B$ y
$\langle 0,0\rangle\leq\langle x,y\rangle\leq\langle a,0\rangle$; pero
al ser $a$ átomo se cumple $x=0$ ó $x=a$ por lo que en realidad se
tiene que $\langle 0,0\rangle=\langle x,y\rangle$ o
$\langle x,y\rangle=\langle a,b\rangle$ y de ello que
$\langle a,b\rangle\in
\operatorname{Atm}(\mathbf{A}\times\mathbf{B})$. Si se da que $a=0$ y
$b\in \operatorname{Atm}(\mathbf{B})$, un razonamiento análogo conduce
a que también
$\langle a,b\rangle\in
\operatorname{Atm}(\mathbf{A}\times\mathbf{B})$. Así pues se tiene
demostrada la afirmación
\hyperref[ex:atm_prod_A]{\ref*{ex:atm_prod_A})}.
\end{solution}

\begin{exercise}
  Considere el álgebra de Boole $\mathbf{F(B_{2},2)}$ y
  \begin{enumerate}
  \item Encuentre sus átomos.
  \item Exprese cada función de $\mathbf{F(B_{2},2)}$ por medio de
    dichos átomos.
  \item Demuestre que los conjuntos: $\{+,\cdot,'\}$, $\{+,'\}$,
    $\{\cdot,'\}$, $\{\supset,0\}$, $\{\uparrow\}$ y 
    $\{\downarrow\}$ son conjuntos suficientes.
  \item Demuestre que valen las siguientes igualdades:
    \begin{align*}
      x+y&=(x\downarrow y)\downarrow (x\downarrow y)\\
      x\supset y&=
         ((x\downarrow x)\downarrow y)\downarrow ((x\downarrow x)\downarrow y) 
    \end{align*}
  \end{enumerate}
\end{exercise}

\begin{solution}
  Sabemos que $\mathbf{F(B_{2},2)}$ es un álgebra de Boole y que
  \begin{equation*}
    \operatorname{card}(B_{2}^{B_{2}\times B_{2}})=2^{4}
  \end{equation*}
  por otra parte, $\mathbf{B_{2}}^{4}$ es también un álgebra de
  Boole con $2^{4}$ elementos. Por tanto:
  \begin{equation*}
    \mathbf{F(B_{2},2)}\cong\mathbf{B_{2}}^{4}
  \end{equation*}
  siendo un isomorfismo entre ambas álgebras la aplicación $\Psi$
  que aplica a cada elemento $f$ de $\mathbf{F(B_{2},2)}$ la
  cuaterna $\langle f(0,0), f(0,1), f(1,0), f(1,1)\rangle$, es
  decir, para todo $f\in\mathbf{F(B_{2},2)}$:
  \begin{equation*}
    \Psi(f)=\langle f(0,0), f(0,1), f(1,0), f(1,1)\rangle
  \end{equation*}
  Sabemos por \hyperref[it:atomosProd]{\ref*{it:atomosProd})} que
  los átomos de $\mathbf{B_{2}}^{4}$ son cuatro, de hecho los
  elementos del conjunto
  \begin{equation*}
    \{\langle 1,0,0,0\rangle,
    \langle 0,1,0,0\rangle,
    \langle 0,0,1,0\rangle,
    \langle 0,0,0,1\rangle\} 
  \end{equation*}
  Por tanto, los átomos de
  $\mathbf{F(B_{2},2)}$ deben ser:
  \begin{align*}
    f_{8}(x,y)&=x'y'\\
    f_{4}(x,y)&=x'y\\
    f_{2}(x,y)&=xy'\\
    f_{1}(x,y)&=xy
  \end{align*}
  De esto deducimos que:
  \begin{align*}
    f_{0}(x,y)&=f_{1}(x,y)f_{2}(x,y)=(xy)(xy')=xyy'\\
    f_{12}(x,y)&=f_{8}(x,y)+f_{4}(x,y)=x'\,y'+x'y
                 =x'(y'+y)=x'\\
    f_{10}(x,y)&=f_{8}(x,y)+f_{2}(x,y)=x'\,y'+xy'
                 =(x'+x)y'=y'\\
    f_{9}(x,y)&=f_{8}(x,y)+f_{1}(x,y)=x'\,y'+xy\\
    f_{6}(x,y)&=f_{4}(x,y)+f_{2}(x,y)=x'y+xy'\\
    f_{5}(x,y)&=f_{4}(x,y)+f_{1}(x,y)=x'y+xy=(x'+x)y=y\\
    f_{3}(x,y)&=f_{2}(x,y)+f_{1}(x,y)=xy'+xy=x(y'+y)=x\\
    f_{14}(x,y)&=f_{8}(x,y)+f_{4}(x,y)+f_{2}(x,y)=f_{12}(x,y)+f_{2}(x,y)
                 =x'+xy'=x'+y'\\
    f_{13}(x,y)&=f_{8}(x,y)+f_{4}(x,y)+f_{1}(x,y)=f_{12}(x,y)+f_{1}(x,y)
                 =x'+xy=x'+y\\
    f_{11}(x,y)&=f_{8}(x,y)+f_{2}(x,y)+f_{1}(x,y)=f_{10}(x,y)+f_{1}(x,y)
                     =y'+xy=x+y'\\
    f_{7}(x,y)&=f_{4}(x,y)+f_{2}(x,y)+f_{1}(x,y)=f_{6}(x,y)+f_{1}(x,y)
                =xy'+y=x+y\\
    f_{15}(x,y)&=f_{8}(x,y)+f_{4}(x,y)+f_{2}(x,y)+f_{1}(x,y)=1
  \end{align*}
  Lo anterior demuestra $\{+,\cdot,'\}$ es un conjunto
  suficiente. Para la demostración de la suficiencia del resto de conjuntos:
  \begin{enumerate}
  \item $\{+,'\}$ es suficiente porque $xy=(x'+y')'$.
  \item $\{\cdot,'\}$ es suficiente porque $x+y=(x'y')'$.
  \item $\{\supset,0\}$ es suficiente porque:
    \begin{align*}
      x'&=x\supset 0\\
      x+y&=(x\supset 0)\supset y\\
      xy&=(x\supset(y\supset 0))\supset 0
    \end{align*}
  \item $\{\uparrow\}$ es suficiente porque:
    \begin{align*}
      x'&=(xx)'\\
        &=x'+x'\\
        &=x\uparrow x\\
      x+y&=x''+y''\\
         &=x'\uparrow y'\\
         &=(x\uparrow x)\uparrow(y\uparrow y)
    \end{align*}
  \item $\{\downarrow\}$ es suficiente porque:
    \begin{align*}
      x'&=(x+x)'\\
        &=x'x'\\
        &=x\downarrow x\\
      xy&=x''y''\\
        &=x'\downarrow y'\\
        &=(x\downarrow x)\downarrow (y\downarrow y)
    \end{align*}
  \end{enumerate}
  Por otra parte:
  \begin{align*}
    x+y&=(x+y)(x+y)\\
       &=(x'y')'(x'y')'\\
       &=(x\downarrow y)\downarrow (x\downarrow y)\\
    x\supset y&=x'+y\\
              &=(xy')'\\
              &=(xy'+xy')'\\
              &=(xy')'(xy')'\\
              &=(x''y')'(x''y')'\\
              &=(x'\downarrow y)\downarrow (x'\downarrow y)\\
              &=((x\downarrow x)\downarrow y)\downarrow
                ((x\downarrow x)\downarrow y)
  \end{align*}
\end{solution}

\begin{exercise}
  Sobre expresiones boolenas $\varphi$ definamos la siguiente
  función $p_{G}$ que propociona porlinomios con coeficientes de
  $\mathbb{Z}_{2}$:
  \begin{equation*}
    \operatorname{p_{G}}(\varphi)=
    \begin{cases}
      0,&\text{ si }\varphi\equiv 0\\
      1,&\text{ si }\varphi\equiv 1\\
      x_{i},&\text{ si }\varphi\equiv x_{i}\\
      \operatorname{p_{G}}(\alpha)+1,
      &\text{ si }\varphi\equiv\alpha'\\
      \operatorname{p_{G}}(\alpha)\operatorname{p_{G}}(\beta)
      +\operatorname{p_{G}}(\alpha)
      +\operatorname{p_{G}}(\beta),&\text{ si }\varphi\equiv\alpha+\beta\\
      \operatorname{p_{G}}(\alpha)\operatorname{p_{G}}(\beta),
      &\text{ si }\varphi\equiv\alpha\cdot\beta
    \end{cases}
  \end{equation*}
  Para cualquier expresión booleana $\varphi$,
  $\operatorname{p_{G}}(\varphi)$ está bien definida (por el
  \textit{principio de lectura única}) y es denominado el
  \href{https://en.wikipedia.org/wiki/Ivan_Ivanovich_Zhegalkin}{\textit{Polinomio
      de Zhegalkine}} de $\varphi$. Encuentre el polinomio de
  Zhegalkin de las siguientes expresiones booleanas:
  \begin{enumerate}
  \item $x\supset y$
  \item $x\uparrow y$ (Barra de Sheffer (Sheffer stroke) o
    $\operatorname{NAND}$, con notación de Bocheński $Dxy$)
  \item $x\downarrow y$ (Daga de Quine (Quine dagger) o Flecha de
    Peirce (Peirce's arrow) o $\operatorname{NOR}$, con notación de Bocheński $Xxy$)
  \item $x\equiv y$
  \item $x\oplus y$
  \item $x\supset(y\supset z)$
  \item $(x\supset y)\supset z$
  \end{enumerate}
\end{exercise}

\begin{solution}
  \begin{align*}
    \operatorname{p_{G}}(x\supset y)&=\operatorname{p_{G}}(x'+y)\\
         &=\operatorname{p_{G}}(x')\operatorname{p_{G}}(y)
           +\operatorname{p_{G}}(x')
           +\operatorname{p_{G}}(y)\\
          &=(x+1)y+(x+1)+y\\
          &=xy+y+x+1+y\\
          &=xy+x+1
  \end{align*}
  
  \begin{align*}
    \operatorname{p_{G}}(x\uparrow y)&=\operatorname{p_{G}}(x'+y')\\
    &=\operatorname{p_{G}}(x')\operatorname{p_{G}}(y')
    +\operatorname{p_{G}}(x')
      \operatorname{p_{G}}(y')\\
    &=(x+1)(y+1)+x+1+y+1\\
    &=xy+x+y+1+x+y\\
    &=xy+1
  \end{align*}

  \begin{align*}
    \operatorname{p_{G}}(x\downarrow y)&=\operatorname{p_{G}}(x'y')\\
           &=\operatorname{p_{G}}(x')\operatorname{p_{G}}(y')\\
           &=(x+1)(y+1)\\
           &=xy+x+y+1
  \end{align*}

  \begin{align*}
    \operatorname{p_{G}}(x\equiv y)&=\operatorname{p_{G}}(x'y'+xy)\\
       &=\operatorname{p_{G}}(x'y')\operatorname{p_{G}}(xy)+\operatorname{p_{G}}(x'y')
        +\operatorname{p_{G}}(xy)\\
       &=(x+1)(y+1)xy+(x+1)(y+1)+xy\\
       &=x+y+1
  \end{align*}
  \begin{align*}
    \operatorname{p_{G}}(x\oplus y)&=\operatorname{p_{G}}(x'y+xy')\\
       &=\operatorname{p_{G}}(x'y)\operatorname{p_{G}}(xy')
        +\operatorname{p_{G}}(x'y)
        +\operatorname{p_{G}}(xy')\\
       &=\operatorname{p_{G}}(x')\operatorname{p_{G}}(y)\operatorname{p_{G}}(x)\operatorname{p_{G}}(y')
        +\operatorname{p_{G}}(x')\operatorname{p_{G}}(y)
        +\operatorname{p_{G}}(x)\operatorname{p_{G}}(y')\\
       &=(x+1)yx(y+1)+(x+1)y+x(y+1)\\
       &=(xy+yx)(y+1)+xy+y+xy+x\\
       &=x+y
  \end{align*}
  El resto del ejercicio es sencillo y análogo a lo anterior.
\end{solution}

\begin{exercise}
  \label{ex:digitos7segmentos}
  % http://www.zeepedia.com/read.php?converting_between_pos_and_sop_using_the_k-map_digital_logic_design&b=9&c=11
  El dígito a 7 segmentos se forma iluminando en la pantalla los
  segmentos apropiados según explica esquemáticamente la
  \hyperref[fg:digitos7segmentos]{Figura \ref*{fg:digitos7segmentos}}
  y así puede mostrar los números decimales del 0 al 9. El dígito
  queda construido por iluminación de hasta los segmentos: a,b,c,d,e,f
  y g, que se activan o desactivan mediante un circuito digital
  dependiendo del número que ha de ser mostrado; por ejemplo, el
  dígito 3 requiera la activación de exáctamente los segmentos:
  a,b,c,d y g. Para mostrar el dígito 7 quedarán desactivados
  exactamente los segmentos: d, e, f y g.

  El circuito que activa los segmentos adecuados para mostrar uno
  cualquiera de los dígitos se conoce como \textit{Decodificador BCD a
    7 segmentos};\index{decodificador BCD a
    7 segmentos} su entrada es un número BCD de 4-bit entre 0 y
  9. Cada una de las 7 salidas del circuito conectan con los siete
  segmentos.

  Para implementar el circuito decodificador de cuatro entradas
  ($n=4$) y siete salidas ($m=7$) debemos hacer una tabla para cada
  segmento, representando su estado para combinación de entradas. Así
  pues debemos determinar siete expresiones, una por cada segmento,
  antes de implementar el circuito.

  Puesto que los números representados por una entrada de 4 bits son
  16, la tabla de cada función tendrá 16 entradas combinacionales. No
  obstante las seis últimas combinaciones son ``no-importa'', ya que
  ninguna de ellas ocurre al ser las entradas de números BCD de 4
  bits. Los estados no-importa ayudarán sin embargo a simplificar las
  expresiones booleanas para los siete segmentos. Según lo que
  explican la: \hyperref[fg:segmentosAaF]{Figura
    \ref*{fg:segmentosAaF}}, \hyperref[fg:KsegmentosAaF]{Figura
    \ref*{fg:KsegmentosAaF}} y
  \hyperref[fg:segmentoG]{\ref*{fg:segmentoG}}, la función booleana
  $D_{BCD7}(a,b,c,d)$ que codifica el \textit{Decodificador BCD a
    7 segmentos} es la siguiente:
  \begin{align*}
    D_{BCD7}(a,b,c,d)=&\langle a+c+bd+\bar{b}\bar{d},\\
                    &\bar{b}+\bar{c}\bar{d}+cd,\\
                    &b+\bar{c}+d,\\
                    &a+\bar{b}\bar{d}+\bar{b}c+c\bar{d}+b\bar{c}d,\\
                    &\bar{b}\bar{d}+c\bar{d},\\
                    &a+b\bar{c}+b\bar{d}+\bar{c}\bar{d},\\
                    &a+b\bar{c}+\bar{b}c+c\bar{d}\rangle
  \end{align*}
  donde las entradas de la 7-upla que representa a $D_{BCD7}(a,b,c,d)$
  corresponden a los segmentos en el orden de la plabra abcdefg.
\end{exercise}

\begin{figure}[!hbp]
  \centering
  \mbox{
    \subcaptionbox{\label{sfg:digitos} dígito a 7 segmentos}{
      \includegraphics[width=.20\textwidth]{digitos.png}
    }
    \qquad
    \subcaptionbox{\label{sfg:digitosSegmentos} segmentos activos por cada dígito}
    {
      \begin{tabular}[b]{|c|l|}
        \hline
        Dígito&\multicolumn{1}{|c|}{Segmentos}\\\hline
        0&a,b,c,d,e,f\\
        1&b,c\\
        2&a,b,d,e,g\\
        3&a,b,c,d,g\\
        4&b,c,f,g\\
        5&a,c,d,f,g\\
        6&a,c,d,e,f,g\\
        7&a,b,c\\
        8&a,b,c,d,e,f,g\\
        9&a,b,c,d,f,g\\\hline
      \end{tabular}
    }
  }
  \caption{\label{fg:digitos7segmentos} Dígitos a 7 segmentos y su
    formación activándolos.}
\end{figure}

\begin{figure}[!hbp]
  \centering
  \mbox{
    \subcaptionbox{\label{sfg:segmentoA} Codif. segmento a}{
      \begin{tabular}[b]{|c|c|c|c|c|}
        \hline
        \multicolumn{4}{|c|}{Input}&\multicolumn{1}{|c|}{Output}\\\hline
        a&b&c&d&segmento a\\\hline
        0&0&0&0&1\\\hline
        0&0&0&1&0\\\hline
        0&0&1&0&1\\\hline
        0&0&1&1&1\\\hline
        0&1&0&0&0\\\hline
        0&1&0&1&1\\\hline
        0&1&1&0&1\\\hline
        0&1&1&1&1\\\hline
        1&0&0&0&1\\\hline
        1&0&0&1&1\\\hline
        1&0&1&0&x\\\hline
        1&0&1&1&x\\\hline
        1&1&0&0&x\\\hline
        1&1&0&1&x\\\hline
        1&1&1&0&x\\\hline
        1&1&1&1&x\\\hline
      \end{tabular}
    }
    \subcaptionbox{\label{sfg:segmentoB} Codif. segmento b}{
      \begin{tabular}[b]{|c|c|c|c|c|}
        \hline
        \multicolumn{4}{|c|}{Input}&\multicolumn{1}{|c|}{Output}\\\hline
        a&b&c&d&segmento b\\\hline
        0&0&0&0&1\\\hline
        0&0&0&1&1\\\hline
        0&0&1&0&1\\\hline
        0&0&1&1&1\\\hline
        0&1&0&0&1\\\hline
        0&1&0&1&0\\\hline
        0&1&1&0&0\\\hline
        0&1&1&1&1\\\hline
        1&0&0&0&1\\\hline
        1&0&0&1&1\\\hline
        1&0&1&0&x\\\hline
        1&0&1&1&x\\\hline
        1&1&0&0&x\\\hline
        1&1&0&1&x\\\hline
        1&1&1&0&x\\\hline
        1&1&1&1&x\\\hline
      \end{tabular}
    }
    \subcaptionbox{\label{sfg:segmentoC} Codif. segmento c}
     {
      \begin{tabular}[b]{|c|c|c|c|c|}
        \hline
        \multicolumn{4}{|c|}{Input}&\multicolumn{1}{|c|}{Output}\\\hline
        a&b&c&d&segmento c\\\hline
        0&0&0&0&1\\\hline
        0&0&0&1&1\\\hline
        0&0&1&0&0\\\hline
        0&0&1&1&1\\\hline
        0&1&0&0&1\\\hline
        0&1&0&1&1\\\hline
        0&1&1&0&1\\\hline
        0&1&1&1&1\\\hline
        1&0&0&0&1\\\hline
        1&0&0&1&1\\\hline
        1&0&1&0&x\\\hline
        1&0&1&1&x\\\hline
        1&1&0&0&x\\\hline
        1&1&0&1&x\\\hline
        1&1&1&0&x\\\hline
        1&1&1&1&x\\\hline
      \end{tabular}
    }
  }
  \mbox{
        \subcaptionbox{\label{sfg:segmentoD} Codif. segmento d}
        {
          \begin{tabular}[b]{|c|c|c|c|c|}
            \hline
            \multicolumn{4}{|c|}{Input}&\multicolumn{1}{|c|}{Output}\\\hline
            a&b&c&d&segmento d\\\hline
            0&0&0&0&1\\\hline
            0&0&0&1&0\\\hline
            0&0&1&0&1\\\hline
            0&0&1&1&1\\\hline
            0&1&0&0&0\\\hline
            0&1&0&1&1\\\hline
            0&1&1&0&1\\\hline
            0&1&1&1&0\\\hline
            1&0&0&0&1\\\hline
            1&0&0&1&1\\\hline
            1&0&1&0&x\\\hline
            1&0&1&1&x\\\hline
            1&1&0&0&x\\\hline
            1&1&0&1&x\\\hline
            1&1&1&0&x\\\hline
            1&1&1&1&x\\\hline
          \end{tabular}
        }
    \subcaptionbox{\label{sfg:segmentoE} Codif. segmento e}
    {
      \begin{tabular}[b]{|c|c|c|c|c|}
        \hline
        \multicolumn{4}{|c|}{Input}&\multicolumn{1}{|c|}{Output}\\\hline
        a&b&c&d&segmento e\\\hline
        0&0&0&0&1\\\hline
        0&0&0&1&0\\\hline
        0&0&1&0&1\\\hline
        0&0&1&1&0\\\hline
        0&1&0&0&0\\\hline
        0&1&0&1&0\\\hline
        0&1&1&0&1\\\hline
        0&1&1&1&0\\\hline
        1&0&0&0&1\\\hline
        1&0&0&1&0\\\hline
        1&0&1&0&x\\\hline
        1&0&1&1&x\\\hline
        1&1&0&0&x\\\hline
        1&1&0&1&x\\\hline
        1&1&1&0&x\\\hline
        1&1&1&1&x\\\hline
      \end{tabular}
    }
    \subcaptionbox{\label{sfg:segmentoF} Codif. segmento f}
    {
      \begin{tabular}[b]{|c|c|c|c|c|}
        \hline
        \multicolumn{4}{|c|}{Input}&\multicolumn{1}{|c|}{Output}\\\hline
        a&b&c&d&segmento f\\\hline
        0&0&0&0&1\\\hline
        0&0&0&1&0\\\hline
        0&0&1&0&0\\\hline
        0&0&1&1&0\\\hline
        0&1&0&0&1\\\hline
        0&1&0&1&1\\\hline
        0&1&1&0&1\\\hline
        0&1&1&1&0\\\hline
        1&0&0&0&1\\\hline
        1&0&0&1&1\\\hline
        1&0&1&0&x\\\hline
        1&0&1&1&x\\\hline
        1&1&0&0&x\\\hline
        1&1&0&1&x\\\hline
        1&1&1&0&x\\\hline
        1&1&1&1&x\\\hline
      \end{tabular}
    }
  }
  \caption{\label{fg:segmentosAaF} Codificación por segmentos de su activación del a al f.}
\end{figure}

\begin{figure}[!hbp]
  \centering
  \mbox{
    \subcaptionbox{\label{sfg:KsegmentoA} Mapa K segmento a: $a+c+bd+\bar{b}\bar{d}$}
    {
      \begin{karnaugh-map}[4][4][1][$cd$][$ab$]
        \manualterms{1,0,1,1,0,1,1,1,1,1,x,x,x,x,x,x}
        \implicantcorner[0,2]
        \implicant{12}{10}
        \implicant{5}{15}
        \implicant{3}{10}
      \end{karnaugh-map}
    }
    \subcaptionbox{\label{sfg:KsegmentoB} Mapa K segmento b: $\bar{b}+\bar{c}\bar{d}+cd$}
    {
      \begin{karnaugh-map}[4][4][1][$cd$][$ab$]
        \manualterms{1,1,1,1,1,0,0,1,1,1,x,x,x,x,x,x}
        \implicantedge{0}{2}{8}{10}
        \implicant{0}{8}
        \implicant{3}{11}
      \end{karnaugh-map}
    }
  }
  \mbox{
    \subcaptionbox{\label{sfg:KsegmentoC} Mapa K segmento c: $b+\bar{c}+d$}
    {
      \begin{karnaugh-map}[4][4][1][$cd$][$ab$]
        \manualterms{1,1,0,1,1,1,1,1,1,1,x,x,x,x,x,x}
        \implicant{0}{9}
        \implicant{1}{11}
        \implicant{4}{14}
      \end{karnaugh-map}
    }
    \subcaptionbox{\label{sfg:KsegmentoD}Mapa K segmento d: $a+\bar{b}\bar{d}+\bar{b}c+c\bar{d}+b\bar{c}d$}
    {
      \begin{karnaugh-map}[4][4][1][$cd$][$ab$]
        \manualterms{1,0,1,1,0,1,1,0,1,1,x,x,x,x,x,x}
        \implicantcorner[0,2]
        \implicant{12}{10}
        \implicantedge{3}{2}{11}{10}
        \implicant{2}{10}
        \implicant{5}{13}
      \end{karnaugh-map}   
    }
  }
  \mbox{
    \subcaptionbox{\label{sfg:KsegmentoE} Mapa K segmento e:
      $\bar{b}\bar{d}+c\bar{d}$}
    {
      \begin{karnaugh-map}[4][4][1][$cd$][$ab$]
        \manualterms{1,0,1,0,0,0,1,0,1,0,x,x,x,x,x,x}
        \implicantcorner[0,2]
        \implicant{2}{10}
      \end{karnaugh-map}
    }
    \subcaptionbox{\label{sfg:KsegmentoF} Mapa K segmento f: $a+b\bar{c}+b\bar{d}+\bar{c}\bar{d}$}{
      \begin{karnaugh-map}[4][4][1][$cd$][$ab$]
        \manualterms{1,0,0,0,1,1,1,0,1,1,x,x,x,x,x,x}
        \implicant{0}{8}
        \implicant{4}{13}
        \implicant{12}{10}
        \implicantedge{4}{12}{6}{14}
      \end{karnaugh-map}
    }
  }
  \caption{\label{fg:KsegmentosAaF} Mapas K para las funciones de
    codificación de la activación de los segmentos a a f}
\end{figure}

\begin{figure}[!hbp]
  \centering
  \mbox{
    \subcaptionbox{\label{sfg:segmentG} Codif. segmento g}
    {
      \begin{tabular}[b]{|c|c|c|c|c|}
      \hline
      \multicolumn{4}{|c|}{Input}&\multicolumn{1}{|c|}{Output}\\\hline
      a&b&c&d&segmento g\\\hline
      0&0&0&0&0\\\hline
      0&0&0&1&0\\\hline
      0&0&1&0&1\\\hline
      0&0&1&1&1\\\hline
      0&1&0&0&1\\\hline
      0&1&0&1&1\\\hline
      0&1&1&0&1\\\hline
      0&1&1&1&0\\\hline
      1&0&0&0&1\\\hline
      1&0&0&1&1\\\hline
      1&0&1&0&x\\\hline
      1&0&1&1&x\\\hline
      1&1&0&0&x\\\hline
      1&1&0&1&x\\\hline
      1&1&1&0&x\\\hline
      1&1&1&1&x\\\hline
      \end{tabular}
    }
    \subcaptionbox{\label{sfg:KsegmentoG} Mapa K segmento g: $a+b\bar{c}+\bar{b}c+c\bar{d}$}
    {
      \begin{karnaugh-map}[4][4][1][$cd$][$ab$]
        \manualterms{0,0,1,1,1,1,1,0,1,1,x,x,x,x,x,x}
        \implicantedge{3}{2}{11}{10}
        \implicant{2}{10}
        \implicant{4}{13}
        \implicant{12}{10}
      \end{karnaugh-map}   
    }
  }
  \caption{\label{fg:segmentoG} Codificación de activación del segmento g y mapa K correspondiente.}
\end{figure}

\begin{exercise}
  Considere la función booleana $f\colon B^{4}\longrightarrow B$
  definida por:
  \begin{equation*}
    f(x,y,z,u)=\sum m(7,11,13,14,15)
  \end{equation*}
  \begin{enumerate}
  \item Calcule una expresión minimal de $f$ a condición de ser
    suma de productos de literales. \label{en:SOP}
  \item Calcule una expresión minimal de $f$ a condición de ser
    producto de sumas de literales. \label{en:POS}
  \item Calcule el coste de cada una de las expresiones encontradas
    para $f$ en los apartados \hyperref[en:POS]{\ref*{en:POS})} y
    \hyperref[en:SOP]{\ref*{en:SOP})} y señale cuál es la del menor.
  \end{enumerate} %del examen de incidencias
\end{exercise}

\begin{solution}
  \begin{figure}
    \centering
    \mbox{
      \subcaptionbox{\label{examen:min} Suma de minterm}{
        \begin{karnaugh-map}[4][4][1][$zu$][$xy$]
          \minterms{7,11,13,14,15}
          \autoterms[0]
          \implicant{7}{15}
          \implicant{13}{15}
          \implicant{15}{14}
          \implicant{15}{11}
        \end{karnaugh-map}
      }
      \qquad
      \subcaptionbox{\label{examen:max} Producto de maxterm}{
        \begin{karnaugh-map}[4][4][1][$zu$][$xy$]
          \maxterms{0,1,2,3,4,5,6,8,9,10,12}
          \autoterms[1]
          \implicant{0}{5}
          \implicant{0}{2}
          \implicant{0}{8}
          \implicantedge{0}{4}{2}{6}
          \implicantedge{0}{1}{8}{9}
          \implicantcorner[0,2]
        \end{karnaugh-map}
      }
    }
    \caption{\label{examen:total} Expresiones de $f$}
  \end{figure}
  Para la solución del problema nos basaremos en los mapas K de la
  \hyperref[examen:total]{Figura \ref*{examen:total}}. Del mapa K de
  la \hyperref[examen:min]{subfigura \ref*{examen:min})} deducimos:
  \begin{equation}
    \label{eq:ex1}
    f(x,y,z,u)=yzu+xyu+xzu+xyz
  \end{equation}
  y del mapa K de
  la \hyperref[examen:max]{subfigura \ref*{examen:max})} deducimos:
  \begin{equation}
    \label{eq:ex2}
    f(x,y,z,u)=(x+y)(z+u)(y+u)(x+z)(y+z)(x+u)
  \end{equation}
  El análisis de los costes es el siguiente:
  \begin{itemize}
  \item Expresión \hyperref[eq:ex1]{(\ref*{eq:ex1})}
    \begin{itemize}
    \item \texttt{and}: 4
    \item \texttt{or}: 1
    \item entradas: 16
    \end{itemize}
    que arroja un total de 21
  \item Expresión \hyperref[eq:ex2]{(\ref*{eq:ex2})}
    \begin{itemize}
    \item \texttt{and}: 1
    \item \texttt{or}: 6
    \item entradas: 18
    \end{itemize}
    que arroja un total de 25
  \end{itemize}
\end{solution}

\begin{exercise}
Sean $m$ y $n$ números naturales no nulos tales que
  $\mathbf{D(m)}$ y $\mathbf{D(n)}$ son álgebras de Boole no
  triviales. Demuestre que son equivalentes las siguientes
  afirmaciones:
  \begin{enumerate}
  \item $\mathbf{D(mn)}$ es un álgebra de Boole.
  \item $(m,n)=1$.
  \end{enumerate}
  Demuestre además que si $m$ y $n$ son números naturales no nulos
  tales que $\mathbf{D(m)}$, $\mathbf{D(n)}$ y $\mathbf{D(mn)}$ son
  álgebras de Boole, entonces $\mathbf{D(m)}\times\mathbf{D(n)}$ y
  $\mathbf{D(mn)}$ son álgebras de Boole isomorfas.
\end{exercise}

\begin{solution}
  Sean $m$ y $n$ números naturales no nulos tales que $\mathbf{D(m)}$
  y $\mathbf{D(n)}$ son álgebras de Boole no triviales. Supongamos
  $\mathbf{D(mn)}$ es un álgebra de Boole y que es isomorfa a
  $\mathbf{D(m)}\times\mathbf{D(n)}$; en tal caso $\mathbf{D(mn)}$ es
  un álgebra de Boole. Por tanto, en la factorización de $mn$ como
  producto de números primos no figura ningún primo elevado a una
  potencia superior a $1$; de donde $m$ y $n$ no tienen ningún factor
  primo en común, o sea, $(m,n)=1$. Recíprocamente, supongamos que
  $(m,n)=1$. Por la hipótesis general $P(m)$ y $P(n)$ valen; donde
  $P(r)$ es la afirmación ``existe un número natural $k_{r}$ tal que:
  $r=\prod_{i=0}^{k_{r}}p_{i}$; para todo $0\leq i\leq k_{r}$, $p_{i}$
  es primo y $p_{i}\neq p_{j}$, siempre que $0\leq i<j\leq
  k_{r}$''. Si $(m,n)=1$, $P(mn)$ vale; por lo que $\mathbf{D(mn)}$ es
  un álgebra de Boole. Por otra parte, supongamos que $\mathbf{D(m)}$
  y $\mathbf{D(n)}$ son álgebras de Boole no triviales. Entonces:
  \begin{itemize}
  \item Los átomos del álgebra de Boole
    $\mathbf{D(m)}\times\mathbf{D(n)}$ son los elementos de
    \begin{equation*}
      \{\langle p,1\rangle\colon p\text{ es primo y }p\mid m\}
      \cup
      \{\langle 1,q\rangle\colon p\text{ es primo y }q\mid n\}
    \end{equation*}
    por lo que posee $k_{m}+k_{n}+2$ átomos.
  \item En las hipótesis, $\mathbf{D(mn)}$ es un álgebra de Boole no
    trivial y su número de átomos es también $k_{m}+k_{n}+2$.
  \end{itemize}
  Al ser $\mathbf{D(m)}\times\mathbf{D(n)}$ y $\mathbf{D(mn)}$
  álgebras de Boole no triviales y tener el mismo número de átomos,
  deben ser isomorfas. Pero si alguna de ellas, o las dos, son
  triviales es resultado es a su vez evidente.
\end{solution}

\begin{exercise}
  Considere la función booleana $f\colon B^{4}\longrightarrow B$
  definida por:
  \begin{equation*}
    f(x,y,z,u)=\sum m(0,1,3,4,7,11,13,15)+\sum d(9,12,14)
  \end{equation*}
  \begin{enumerate}
  \item Calcule una expresión minimal de $f$ a condición de ser
    suma de productos de literales. \label{en:SOP1}
  \item Calcule una expresión minimal de $f$ a condición de ser
    producto de sumas de literales. \label{en:POS1}
  \item Calcule el coste de cada una de las expresiones que se piden
    para $f$ en los apartados \hyperref[en:POS1]{\ref*{en:POS1})} y
    \hyperref[en:SOP1]{\ref*{en:SOP1})} y compare los costos.
  \item De una expresión de $f$ con coste menor que el de cualquiera
    de las anteriores.
  \end{enumerate} %lmd_giim_20172018_ordinario.tex
\end{exercise}

\begin{solution}
  Para la solución del problema nos basaremos en los mapas K de la
  \hyperref[examen:total1]{Figura \ref*{examenExt:total1}}. Del mapa K de
  la \hyperref[examen:min1]{subfigura \ref*{examenExt:min1})} deducimos:
  \begin{equation}
    \label{eq:exext11}
    f(x,y,z,u)=x'z'u'+y'u+xu+zu
  \end{equation}
  y del mapa K de
  la \hyperref[examen:max1]{subfigura \ref*{examenExt:max1})} deducimos:
  \begin{equation}
    \label{eq:exext21}
    f(x,y,z,u)=(x'+u)(z'+u)(x+y'+z+u')
  \end{equation}
  El análisis de los costos es el siguiente:
  \begin{itemize}
  \item Expresión \hyperref[eq:exext11]{(\ref*{eq:exext11})}
    \begin{itemize}
    \item \texttt{and}: 1
    \item \texttt{or}: 4
    \item entradas: 13
    \end{itemize}
    que arroja un total de 18
  \item Expresión \hyperref[eq:exext21]{(\ref*{eq:exext21})}
    \begin{itemize}
    \item \texttt{and}: 1
    \item \texttt{or}: 3
    \item entradas: 11
    \end{itemize}
    que arroja un total de 15
  \end{itemize}
  No obstante:
  \begin{equation*}
    f(x,y,z,u)=x'z'u'+(y'+x+z)u
  \end{equation*}
  cuyo costo es $14$, pues
  \begin{itemize}
    \item \texttt{and}: 2
    \item \texttt{or}: 2
    \item entradas: 10
    \end{itemize}
    Sin embargo, esta solución ni es POS ni SOP.
    \begin{figure}
      \centering
    \mbox{
      \subcaptionbox{\label{examenExt:min1} Suma de minterm}{
        \begin{karnaugh-map}[4][4][1][$zu$][$xy$]
          \minterms{0,1,3,4,7,11,13,15}
          \indeterminants{9,12,14}
          \autoterms[0]
          \implicantedge{1}{3}{9}{11}
          \implicant{0}{4}
          \implicant{3}{11}
          \implicant{13}{11}
        \end{karnaugh-map}
      }
      \qquad
      \subcaptionbox{\label{examenExt:max1} Producto de maxterm}{
        \begin{karnaugh-map}[4][4][1][$zu$][$xy$]
          %\minterms{0,1,3,4,7,11,13,15}
          \maxterms{2,5,6,8,10}
          \indeterminants{9,12,14}
          %\autoterms[0]
          \autoterms[1]
          \implicant{2}{10}
          \implicant{5}{5}
          \implicantedge{12}{8}{14}{10}
        \end{karnaugh-map}
      }
    }
    \caption{\label{examenExt:total1} Expresiones de $f$}
  \end{figure}
\end{solution}
  
\begin{exercise}
  Considere la función booleana $f\colon B^{4}\longrightarrow B$
  definida por:
  \begin{equation*}
    f(x,y,z,u)=\sum m(7,9,10,11,12,13,14,15)
  \end{equation*}
  \begin{enumerate}
  \item Calcule una expresión minimal de $f$ a condición de ser
    suma de productos de literales. \label{en:SOP}
  \item Calcule una expresión minimal de $f$ a condición de ser
    producto de sumas de literales. \label{en:POS}
  \item Calcule el coste de cada una de las expresiones que se piden
    para $f$ en los apartados \hyperref[en:POS]{\ref*{en:POS})} y
    \hyperref[en:SOP]{\ref*{en:SOP})} y compare los costos.
  \end{enumerate} %del examen de incidencias
\end{exercise}

\begin{solution}
  \begin{figure}
    \centering
    \mbox{
      \subcaptionbox{\label{examenExt:min} Suma de minterm}{
        \begin{karnaugh-map}[4][4][1][$zu$][$xy$]
          \minterms{7,9,10,11,12,13,14,15}
          \autoterms[0]
          \implicant{7}{15}
          \implicant{12}{14}
          \implicant{13}{11}
          \implicant{15}{10}
        \end{karnaugh-map}
      }
      \qquad
      \subcaptionbox{\label{examenExt:max} Producto de maxterm}{
        \begin{karnaugh-map}[4][4][1][$zu$][$xy$]
          \maxterms{0,1,2,3,4,5,6,8}
          \autoterms[1]
          \implicant{0}{2}
          \implicant{0}{5}
          \implicantedge{0}{4}{2}{6}
          \implicantedge{0}{0}{8}{8}
          % \implicantcorner[0,2]
        \end{karnaugh-map}
      }
    }
    \caption{\label{examenExt:total} Expresiones de $f$}
  \end{figure}
  Para la solución del problema nos basaremos en los mapas K de la
  \hyperref[examen:total]{Figura \ref*{examenExt:total}}. Del mapa K de
  la \hyperref[examen:min]{subfigura \ref*{examenExt:min})} deducimos:
  \begin{equation}
    \label{eq:exext1}
    f(x,y,z,u)=xy+xz+xu+yzu
  \end{equation}
  y del mapa K de
  la \hyperref[examen:max]{subfigura \ref*{examenExt:max})} deducimos:
  \begin{equation}
    \label{eq:exext2}
    f(x,y,z,u)=(x+y)(x+z)(x+u)(y+z+u)
  \end{equation}
  El análisis de los costos es el siguiente:
  \begin{itemize}
  \item Expresión \hyperref[eq:exext1]{(\ref*{eq:exext1})}
    \begin{itemize}
    \item \texttt{and}: 4
    \item \texttt{or}: 1
    \item entradas: 13
    \end{itemize}
    que arroja un total de 18
  \item Expresión \hyperref[eq:exext2]{(\ref*{eq:exext2})}
    \begin{itemize}
    \item \texttt{and}: 1
    \item \texttt{or}: 4
    \item entradas: 13
    \end{itemize}
    que arroja un total de 18
  \end{itemize}
  Ambas expresiones tiene el mismo costo.
\end{solution}

\begin{exercise}
  Sea la expresión booleana
  \begin{equation*}
    \varphi\equiv (x'y)\uparrow(x\supset(z\downarrow y'))
  \end{equation*}
  y $f(x,y,z)$ la función booleana asociada a $\varphi$. Encuentre:
  \begin{enumerate}
  \item $p_{G}(\varphi)$
  \item La forma normal disyuntiva perfecta de $f$ (expresión de $f$
    como suma de minterm) 
  \item La forma normal conjuntiva perfecta de $f$ (expresión de $f$
    como producto de maxterm)
  \item Una expresión minimal a condición de ser SOP.
  \end{enumerate}
\end{exercise}

\begin{solution}~
  \begin{enumerate}
  \item Calculemos $\operatorname{p_{G}}(\varphi)$:
    \begin{align*}
      \operatorname{p_{G}}(\varphi)&
          =\operatorname{p_{G}}((x'y)\uparrow(x\supset(z\downarrow y')))\\
        &=\operatorname{p_{G}}(x'y)\operatorname{p_{G}}(x\supset(z\downarrow y'))+1\\
        &=(x+1)y(x \operatorname{p_{G}}(z\downarrow y')+x+1)+1\\
        &=(x+1)y(x(z(y+1)+z+(y+1)+1)+x+1)+1\\
        &=(x+1)y(x(zy+z+z+y+1+1)+x+1)+1\\
        &=(x+1)y(xyz+xy+x+1)+1\\
        &=xyz+xy+xy+xy+xyz+xy+xy+y+1\\
        &=xy+y+1\\
        &=\operatorname{p_{G}}(y\supset x)
    \end{align*}
    y así pues:
    \begin{equation*}
      (x'y)\uparrow(x\supset(z\downarrow y'))=y\supset x=x+y'
    \end{equation*}
    Deducimos que $f$ no depende de $z$ y que $x+y'$ es una expresión
    minimal como suma de productos, pues es SOP y no admite
    simplificación alguna (confírmelo con un mapa K). Calculemos ahora
    las formas normales de $f$:
    \begin{align*}
      f(x,y,z)&=x+y'\\
              &=x\cdot 1+y'\cdot 1\\
              &=x(y+y')+y'(x+x')\\
              &=xy+xy'+xy'+x'y'\\
              &=xy+xy'+x'y'\\
              &=xy(z+z')+xy'(z+z')+x'y'(z+z')\\
              &=x'y'z'+x'y'z+xy'z'+xy'z+xyz'+xyz\\
              &=\sum m(0,1,4,5,6,7)
    \end{align*}
    Así pues,
    \begin{align*}
      f(x,y,z)&=\prod M(2,3)\\
              &=(x+y'+z)(x+y'+z')
    \end{align*}
    Como observación al margen: al ser $f(0,0,0)=1$, sabemos que
    $\operatorname{p_{G}(\varphi)}(0,0,0)=1$; así pues, sin necesidad
    del cálculo $de \operatorname{p_{G}}(\varphi)$, sabemos que el
    término independiente del polinomio de Zhegalkine debe ser igual a
    $1$.
  \end{enumerate}
\end{solution}

\begin{exercise}
  Mediante el algoritmo de Quine-McCluskey (y no otro, ahora)
  encuentre razonadamente todas las expresiones minimales de la
  función:
  \begin{equation*}
    f(x,y,z,u)=\sum m(4,5,7,12,14,15)
  \end{equation*}
  a condición de estar éstas expresadas como suma de
  productos. Determine al menos una expresión de $f$ que teniendo igual 
  costo que cualquiera de las expresiones minimales encontradas, sin embargo
  no esté expresada como suma de productos.
\end{exercise}
  
  \begin{solution}~
    \begin{enumerate}
    \item \textbf{Generación de implicantes primos};  la tabla de la
    \hyperref[fg:exp21819A]{Figura \ref*{fg:exp21819A}} recoge la
    marcha del proceso y los resultados.
        \begin{figure}[!hbp]
          \centering
          \begin{tabular}[c]{rcccrcc}
            \multicolumn{3}{c}{columna 1}
            &
            &\multicolumn{3}{c}{columna 2}
            \\\cline{1-3}\cline{5-7}
            4&0100&$\checkmark$&&\{4,5\}&010{\_}&*\\\cline{1-3}
            5&0101&$\checkmark$&&\{4,12\}&{\_}100&*\\\cline{5-7}
            12&1100&$\checkmark$&&\{5,7\}&01{\_}1&*\\\cline{1-3}
            7&0111&$\checkmark$&&\{12,14\}&11{\_}0&*\\\cline{5-7}
            14&1110&$\checkmark$&&\{7,15\}&{\_}111&*\\\cline{1-3}
            15&1111&$\checkmark$&&\{14,15\}&111{\_}&*\\\cline{5-7}
          \end{tabular}
          \caption{\label{fg:exp21819A} Generación de implicantes primos}
        \end{figure}
      \item \textbf{Construcción de la tabla de implicantes primos};
        es la que aparece en la \hyperref[fg:exp21819B]{Figura
          \ref*{fg:exp21819B}}, donde no aparece ningún implicante
        primo esencial ni columnas ni filas dominadoras.
        \begin{figure}[!hbp]
          \centering
          \begin{tabular}[c]{crl|ccccccccc|}
            &         &               &4&5&7&12&14&15\\\hline
            &\{4,5\}  &010{\_}&$\circ$&$\circ$&&&&\\
            &\{4,12\} &{\_}100&$\circ$&&&$\circ$&&\\
            &\{5,7\}  &01{\_}1&&$\circ$&$\circ$&&&\\
            &\{12,14\}&11{\_}0&&&&$\circ$&$\circ$&\\
            &\{7,15\} &{\_}111&&&$\circ$&&&$\circ$\\
            &\{14,15\}&111{\_}&&&&&$\circ$&$\circ$\\\hline
            &         &              &&&&&&\\\cline{4-9}
          \end{tabular}
          \caption{\label{fg:exp21819B} Tabla de implicantes primos}
        \end{figure}
      \item \textbf{Resolución de la tabla de implicantes primos}; se
        puede llevar a cabo por el algoritmo de
        Petrick. \index{algoritmo ! de Petrick} Encontramos la
        situación de la siguiente tabla:
    \begin{center}
      \begin{tabular}[c]{c|c}
        para cubrir a & basta usar\\\hline
         4&$p_{1}$ ó $p_{2}$\\
         5&$p_{1}$ ó $p_{3}$\\
         7&$p_{3}$ ó $p_{5}$\\
         12&$p_{2}$ ó $p_{4}$\\
        14&$p_{4}$ ó $p_{6}$\\
        15&$p_{5}$ ó $p_{6}$
      \end{tabular}
    \end{center}
      lo que nos lleva a formular la frase $P$ del metalenguaje que
    expresa el modo de cubrir la totalidad de los los minterms como
    una expresión booleana según:
    \begin{equation*}
      (P_{1}+P_{2})(P_{1}+P_{3})(P_{3}+P_{5})(P_{2}+P_{4})(P_{4}+P_{6})
              (P_{5}+P_{6})
    \end{equation*}
    que debemos transformar en suma de productos:
    \begin{align*}
      P&=(P_{1}+P_{2})
           (P_{1}+P_{3})
           (P_{3}+P_{5})
           (P_{2}+P_{4})
           (P_{4}+P_{6})
           (P_{5}+P_{6})&&\\
       &=(P_{1}+P_{2}P_{3})
         (P_{4}+P_{2}P_{6})
         (P_{5}+P_{3}P_{6})&&\\
       &=(P_{1}+P_{2}P_{3})(P_{4}+P_{2}P_{6})P_{5}
       +(P_{1}+P_{2}P_{3})(P_{4}+P_{2}P_{6})P_{3}P_{6}&&\\
       &=(P_{1}+P_{2}P_{3})P_{4}P_{5}
       +(P_{1}+P_{2}P_{3})P_{2}P_{5}P_{6}&&\\
       &+(P_{1}+P_{2}P_{3})P_{4}P_{3}P_{6}
       +(P_{1}+P_{2}P_{3})P_{2}P_{3}P_{6}&&\\
       &=P_{1}P_{4}P_{5}
       +P_{2}P_{3}P_{4}P_{5}
       +P_{1}P_{2}P_{5}P_{6}
       +P_{2}P_{3}P_{5}P_{6}&&\\
       &+P_{1}P_{3}P_{4}P_{6}
       +P_{2}P_{3}P_{4}P_{6}
       +P_{1}P_{2}P_{3}P_{6}
       +P_{2}P_{3}P_{6}&&
    \end{align*}
    donde para simplificar hemos usado $(x+y)(x+z)=x+yz$ (ley
    distributiva) y $x+xy=x$ (ley de absorción). Por tanto, para dar
    la expresión mínima que buscamos nos basaremos en:
    \begin{center}
      \begin{tabular}[c]{|c|c|c|c|}
        pseudónimo&implicante&patrón&representa\\\hline
        $p_{1}$&\{4,5\}  &010{\_}&$\bar{x}y\bar{z}$\\
        $p_{2}$&\{4,12\} &{\_}100&$y\bar{z}\bar{u}$\\
        $p_{3}$&\{5,7\}  &01{\_}1&$\bar{x}yu$\\
        $p_{4}$&\{12,14\}&11{\_}0&$xy\bar{u}$\\
        $p_{5}$&\{7,15\} &{\_}111&$yzu$\\
        $p_{6}$&\{14,15\}&111{\_}&$xyz$\\\hline
      \end{tabular}
      \end{center}
      y será cualquiera de las siguientes 2 opciones:
      \begin{center}
        \begin{tabular}[c]{|ccc|c|}
          \multicolumn{3}{|c|}{implicantes}&$f(x,y,z,u)$\\[1mm]\hline
          $p_{1}$&$p_{4}$&$p_{5}$&$\bar{x}y\bar{z}+xy\bar{u}+yzu$\\
          $p_{2}$&$p_{3}$&$p_{6}$&$y\bar{z}\bar{u}+\bar{x}yu+xyz$\\
          \hline
        \end{tabular}
      \end{center}
      El costo de cualquiera de estas expresiones de $f$ es:
      \begin{itemize}
      \item +\ : 1
      \item $\cdot$\ : 3
      \item entradas: 12
      \item total: 16
      \end{itemize}
      También $f(x,y,z,u)=y(\bar{x}\bar{z}+x\bar{u}+zu)$ que vuelve a
      tener costo $16$.
  \end{enumerate}
  \end{solution}