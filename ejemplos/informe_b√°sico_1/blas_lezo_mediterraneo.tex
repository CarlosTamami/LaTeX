%\chapter{En El Mediterráneo}

\section{Jefe de la Escuadra del Mediterráneo}

Estuvo inactivo en Cádiz un año, hasta que el 3 de noviembre de 1731
se lo nombró jefe de la escuadra naval del Mediterráneo. Esta contaba
con tres navíos de línea, entre ellos el Real Familia, de sesenta
cañones y almiranta de Lezo. La escuadra tenía un papel fundamental en
las ambiciones políticas del rey, que deseaba recuperar los
territorios perdidos en la península itálica en los
tratados\index{Tratado} de paz de la guerra de sucesión. En
reconocimiento de sus servicios al rey, este le concedió en 1731 como
estandarte para su capitana la bandera morada con el escudo de armas
de Felipe V, la Orden del Espíritu Santo ---máxima condecoración
francesa--- y la Orden del Toisón de Oro ---más alta condecoración
española--- alrededor y cuatro anclas en sus extremos.

\section{Primeras Misiones en Italia}

Su primera misión fue participar en diciembre de ese año en la escolta
del infante Carlos, que pasaba a Italia a adueñarse de los ducados de
Parma, Toscana y Plasencia. Lezo mandaba una escuadra de veinticinco
navíos, parte de una flota mayor en la que participaban los ingleses.

Al demorarse los genoveses en devolver los dos millones de pesos
pertenecientes a la Hacienda española que se hallaban depositados en
el Banco de San Jorge, Patiño ordenó a Lezo partir a la capital de la
república para reclamarlos. Lezo ancló en aquel puerto con seis navíos
y exigió un inaudito homenaje a la bandera real de España y la
devolución inmediata del dinero. Sus seis buques apuntaban los cañones
al palacio Doria, como amenaza al Senado de la ciudad.  Mostrando el
reloj de las guardias a los comisionados de la ciudad, que buscaban el
modo de eludir la cuestión del pago, fijó un plazo, transcurrido el
cual la escuadra rompería el fuego contra la ciudad.  De los dos
millones de pesos recibidos, medio millón fue entregado al infante don
Carlos y el resto fue remitido a Alicante para sufragar los gastos de
la expedición que se alistaba para la conquista de Orán.

\section{Expedición a Orán}

En junio de 1732, volvió de Cádiz a Alicante para sumarse a esta
expedición.\footnote{Artículo principal:
  \href{https://is.gd/wPfH3B}{Expedición española a Orán}.} El objetivo
de esta era recuperar la plaza, que había estado en manos españolas
desde 1509 hasta 1709, cuando se había perdido durante la guerra de
sucesión. Retomarla era una cuestión de prestigio para la Corona y
un modo de demostrar el renovado poderío militar y naval español con
la nueva dinastía. Lezo quedó como lugarteniente del capitán de la
flota de la expedición, Francisco Cornejo, mientras que José Carrillo
de Albornoz, conde de Montemar, mandaba las tropas de tierra. Lezo
participó en la operación a bordo del Santiago, parte de la flota de
doce navíos de guerra, dos fragatas, dos bombardas, siete galeras,
dieciocho galeotas, doce barcos varios y más de quinientos transportes
que componían la escuadra de la expedición.

El asedio de Orán comenzó el 29 de junio, con el desembarco de los
veintiséis mil hombres de Montemar. Tras varios choques, se apoderaron
de la plaza el 1 de julio. Sofocadas las últimas resistencias, que
habían costado más bajas que la conquista de la ciudad, la expedición
regresó a España el 1 de agosto, dejando una guarnición. El 2 de
septiembre, Lezo estaba de vuelta en Cádiz.

Cuando la expedición marchó creyendo cumplida su meta, Bey Hassan,
señor de Orán hasta la reconquista española, logró reunir tropas,
aliarse con el bey de Argel y sitiarla. Bombardeó el castillo de
Mazalquivir y aplastó una salida de los defensores, en la que
perecieron más de mil quinientos soldados y además murió el gobernador
español, Álvaro Navia Osorio y Vigil. Este aristócrata fue el autor de
``Reflexiones militares'', libro de cabecera de Federico el Grande.
Ante la desesperada situación de la plaza, el 13 de noviembre se
ordenó a Lezo socorrerla. Este partió de inmediato con los barcos
que estaban listos para realizar la travesía: dos navíos de línea,
cinco menores y veinticinco transportes, que llevaban cinco mil
soldados de refuerzo a la guarnición. Tras dos días de navegación
alcanzó Orán, desbarató el acoso de las nueve galeras argelinas, que
se retiraron al llegar la escuadra española y abasteció a la
guarnición.

Decidido a acabar con la amenaza que suponía la flota argelina,
decidió perseguirla. En febrero de 1733 logró finalmente localizar la
capitana de sesenta cañones, que se refugió en la bahía de Mostagán,
defendida por dos castillos fortificados. Ello no arredró a Lezo, que
entró en la bahía tras la nave argelina despreciando el fuego de los
fuertes, logró poner en fuga una galeaza que surgió inesperadamente
para auxiliar a la galera, abordarla, incendiarla y, a continuación,
destruir los castillos. Retornó entonces primero a Orán y luego
Barcelona, donde recogió cuatro regimientos de infantería que trasladó
a África. Luego reanudó la patrulla de la zona, entre Tetuán y Túnez
durante dos meses, hasta que una epidemia que se desató en la escuadra
lo forzó a regresar a la ciudad de Cádiz.

\section{Último Periodo en Cádiz}

Hasta 1737, mantuvo un continuo litigio con el virrey de Perú por el
sueldo que se le adeudaba, que este se negó hasta entonces a pagarle,
aduciendo falta de fondos. Lezo, empero, no pasó apuros económicos,
tanto por la fortuna de su mujer como por los ingresos que obtuvo de
diversos negocios, entre ellos el comercio en plata, oro y esclavos,
que había realizado mediante un representante durante su estancia en
el Perú. Parte de las ganancias las invertía en rentables pagarés y
deuda; a pesar de sus continuos combates con los ingleses, mantuvo una
cuenta en un banco londinense.

El 6 de junio de 1734 ascendió a teniente general de la Armada y se lo
nombró comandante general del Departamento de Cádiz. Tras realizar una
visita a Madrid, dos años más tarde, en 1736, se lo trasladó a El
Puerto de Santa María como comandante general de los galeones,
responsable de la seguridad del comercio transatlántico. Se dispuso a
preparar la escuadra que escoltó a la última Armada de los Galeones
\index{armada!de los galeones} de
la carrera de Indias, la de 1737. Los preparativos se retrasaron tanto
por las distintas dificultades ---aprestar los buques de guerra,
reclutar las tripulaciones, asegurar el matalotaje, etc., que Lezo
suscitó el disgusto de Patiño, que le intimó que los acelerará.
Cuando por fin la flota estuvo lista en noviembre de 1736, tuvo que
esperar a que los barcos mercantes cargasen las mercancías y no pudo
partir hasta el 3 de febrero de 1737. El convoy, formado por ocho
mercantes, dos navíos de registro y los dos navíos de escolta de Lezo,
realizó la travesía sin contratiempos y arribó a Cartagena de
Indias. La familia de Lezo ---por entonces, formada por su esposa y
seis hijos, ya que uno había fallecido--- permaneció en El Puerto de
Santa María y no acompañó al marino a su nuevo destino en América.