\documentclass[a4paper,openright, spanish,12pt]{article}
\usepackage{babel}
\usepackage[utf8]{inputenc}
\usepackage{enumitem} % enumerados

% a4large.sty -- fill an A4 (210mm x 297mm) page
% Note: 1 inch = 25.4 mm = 72.27 pt
%       1 pt   =  3.5 mm (approx)

% vertical page layout -- one inch margin top and bottom
\topmargin        0   mm    % top margin less 1 inch
\headheight       0   mm    % height of box containing the head
\headsep          0   mm    % space between the head and the body of the page
\textheight     240   mm    % the height of text on the page
\footskip         7   mm    % distance from bottom of body to bottom of foot

% horizontal page layout -- one inch margin each side
\oddsidemargin    0   mm    % inner margin less one inch on odd pages
\evensidemargin   0   mm    % inner margin less one inch on even pages
\textwidth      159.2 mm    % normal width of text on page

\setlength{\parindent}{0em}
%\setlength{\parindent}{4em}
\setlength{\parskip}{1em}
%\renewcommand{\baselinestretch}{1.5}

\newcommand{\HRule}{\rule{\linewidth}{1mm}}

\title
{
\HRule
\begin{flushright}
\Huge
\textbf{La Enumeración}\\[3mm]
\Large
\textbf{con}\\
\Huge
\texttt{enumitem}
\end{flushright}
\HRule 
} 
\author{\large Fco. M. García}

%\date{24/01/2016}

\begin{document}

\maketitle
\tableofcontents % Hace el índice de contenidos.

\section{Detallado}

Qué visitar en Cáceres:
\begin{itemize}
  \item Ciudad vieja.
  \item Guadalupe.
  \item Trujillo.
  \item Plasencia.
\end{itemize}

Qué visitar en Cáceres:
\begin{itemize}[leftmargin=20mm]
  \item Ciudad vieja.
  \item Guadalupe.
  \item Trujillo.
  \item Plasencia.
\end{itemize}

\section{Enumeración}

  Primera lista de la compra:
  \begin{enumerate}
  \item Manzanas.
  \item Plátanos.
  \item Fresas.
  \end{enumerate}
  Segunda lista de la compra:
  \begin{enumerate}[resume]
  \item Limones.
  \item Naranjas.
  \item Pomelos.
  \end{enumerate}  

\end{document}